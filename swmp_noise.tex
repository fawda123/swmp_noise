\documentclass[letterpaper,12pt,oneside]{article}\usepackage[]{graphicx}\usepackage[]{color}
%% maxwidth is the original width if it is less than linewidth
%% otherwise use linewidth (to make sure the graphics do not exceed the margin)
\makeatletter
\def\maxwidth{ %
  \ifdim\Gin@nat@width>\linewidth
    \linewidth
  \else
    \Gin@nat@width
  \fi
}
\makeatother

\definecolor{fgcolor}{rgb}{0.345, 0.345, 0.345}
\newcommand{\hlnum}[1]{\textcolor[rgb]{0.686,0.059,0.569}{#1}}%
\newcommand{\hlstr}[1]{\textcolor[rgb]{0.192,0.494,0.8}{#1}}%
\newcommand{\hlcom}[1]{\textcolor[rgb]{0.678,0.584,0.686}{\textit{#1}}}%
\newcommand{\hlopt}[1]{\textcolor[rgb]{0,0,0}{#1}}%
\newcommand{\hlstd}[1]{\textcolor[rgb]{0.345,0.345,0.345}{#1}}%
\newcommand{\hlkwa}[1]{\textcolor[rgb]{0.161,0.373,0.58}{\textbf{#1}}}%
\newcommand{\hlkwb}[1]{\textcolor[rgb]{0.69,0.353,0.396}{#1}}%
\newcommand{\hlkwc}[1]{\textcolor[rgb]{0.333,0.667,0.333}{#1}}%
\newcommand{\hlkwd}[1]{\textcolor[rgb]{0.737,0.353,0.396}{\textbf{#1}}}%

\usepackage{framed}
\makeatletter
\newenvironment{kframe}{%
 \def\at@end@of@kframe{}%
 \ifinner\ifhmode%
  \def\at@end@of@kframe{\end{minipage}}%
  \begin{minipage}{\columnwidth}%
 \fi\fi%
 \def\FrameCommand##1{\hskip\@totalleftmargin \hskip-\fboxsep
 \colorbox{shadecolor}{##1}\hskip-\fboxsep
     % There is no \\@totalrightmargin, so:
     \hskip-\linewidth \hskip-\@totalleftmargin \hskip\columnwidth}%
 \MakeFramed {\advance\hsize-\width
   \@totalleftmargin\z@ \linewidth\hsize
   \@setminipage}}%
 {\par\unskip\endMakeFramed%
 \at@end@of@kframe}
\makeatother

\definecolor{shadecolor}{rgb}{.97, .97, .97}
\definecolor{messagecolor}{rgb}{0, 0, 0}
\definecolor{warningcolor}{rgb}{1, 0, 1}
\definecolor{errorcolor}{rgb}{1, 0, 0}
\newenvironment{knitrout}{}{} % an empty environment to be redefined in TeX

\usepackage{alltt}
\usepackage[paperwidth=8.5in,paperheight=11in,top=1in,bottom=1in,left=1in,right=1in]{geometry}
\usepackage{setspace}
\usepackage[colorlinks=true,allcolors=Blue]{hyperref}
\usepackage[usenames,dvipsnames]{xcolor}
\usepackage{indentfirst}
\usepackage{titlesec}
\usepackage{multirow}
\usepackage{booktabs}
\usepackage{graphicx}
\usepackage{verbatim}
\usepackage{rotating}
\usepackage{tabularx}
\usepackage{outlines}
\usepackage{lineno}
\usepackage{array}
\usepackage{times}
\usepackage{cleveref}
\usepackage{acronym}
\usepackage[position=t]{subfig}
\usepackage{paralist}
\usepackage[noae]{Sweave}
\usepackage{natbib}
\usepackage{array}
\usepackage{pdflscape}
\usepackage{bm}
\usepackage{showlabels}
\bibpunct{(}{)}{,}{a}{}{,}

% page margins and section title formatting
\linespread{2}
\setlength{\footskip}{0.5in}
\titleformat*{\section}{\Large\bf\em}
\titleformat*{\subsection}{\singlespace\large\bf}
\titleformat*{\subsubsection}{\singlespace\normalsize\bf\em}
\titlespacing{\section}{0in}{0in}{0in}
\titlespacing{\subsection}{0in}{0in}{0in}
\titlespacing{\subsubsection}{0in}{0in}{0in}

% cleveref options
\crefname{table}{Table}{Tables}
\crefname{figure}{Fig.}{Figs.}
\renewcommand{\figurename}{Fig.}

% aliased citations
\defcitealias{BeckIR}{Beck and Hagy, In review}
\defcitealias{CDMO14}{CDMO 2014}
\defcitealias{NRC00}{NRC 2000}
\defcitealias{RDCT14}{RDCT 2014}

% acronyms
\acrodef{DO}[DO]{dissolved oxygen}
\acrodef{EPA}[EPA]{Environmental Protection Agency}
\acrodef{NERRS}[NERRS]{National Estuarine Research Reserve System}
\acrodef{RMSE}[RMSE]{root mean square error}
\acrodef{SWMP}[SWMP]{System Wide Monitoring Program}
\acrodef{WRTDS}[WRTDS]{weighted regression on time, discharge, and season}

% assorted functions
% for multiple rows in table headers
\newcommand{\head}[2]{\multicolumn{1}{>{\arraybackslash}p{#1}}{#2}}
% for milligrams per litre
\newcommand{\mgl}{mg L$^{-1}$}

% hides (not removes) numbering for section, subsection, etc.
% left indents
\renewcommand{\thesection}{}
\renewcommand{\thesubsection}{}
\renewcommand{\thesubsubsection}{}
\makeatletter
\def\@seccntformat#1{\csname #1ignore\expandafter\endcsname\csname the#1\endcsname\quad}
\let\sectionignore\@gobbletwo
\def\@subseccntformat#1{\csname #1ignore\expandafter\endcsname\csname the#1\endcsname\quad}
\let\subsectionignore\@gobbletwo
\def\@subsubseccntformat#1{\csname #1ignore\expandafter\endcsname\csname the#1\endcsname\quad}
\let\subsubsectionignore\@gobbletwo
\let\latex@numberline\numberline
\def\numberline#1{\if\relax#1\relax\else\latex@numberline{#1}\fi}
\makeatother

% dependent data



%knitr options


\IfFileExists{upquote.sty}{\usepackage{upquote}}{}
\begin{document}

\raggedbottom
\linenumbers
\raggedright
\urlstyle{same}
\setlength{\parindent}{0.5in}
\renewcommand\refname{References \vspace{12pt}}

%%%%%%
% title page
\begin{singlespace}
\title{{\bf {\Large Evaluation and correction of noise related to physical processes in estimates of estuary metabolism}}}
\author{
  {\bf {\normalsize Marcus W. Beck$^1$, Michael C. Murrell$^2$, James D. Hagy III$^2$, Jane M. Caffrey$^3$}}
  \\\\{\textit {\normalsize $^1$ORISE Research Participation Program}}
  \\{\textit {\normalsize USEPA National Health and Environmental Effects Research Laboratory}}
	\\{\textit {\normalsize Gulf Ecology Division, 1 Sabine Island Drive, Gulf Breeze, FL 32561}}
	\\{\textit {\normalsize Phone: 850-934-2480, Fax: 850-934-2401, Email: \href{mailto:beck.marcus@epa.gov}{beck.marcus@epa.gov}}}
  \\\\{\textit {\normalsize $^2$USEPA National Health and Environmental Effects Research Laboratory}}
	\\{\textit {\normalsize Gulf Ecology Division, 1 Sabine Island Drive, Gulf Breeze, FL 32561}}
	\\{\textit {\normalsize Phone: 850-934-2433, Fax: 850-934-2401, Email: \href{mailto:murrell.michael@epa.gov}{murrell.michael@epa.gov}}}
  \\\\{\textit {\normalsize $^3$USEPA National Health and Environmental Effects Research Laboratory}}
	\\{\textit {\normalsize Gulf Ecology Division, 1 Sabine Island Drive, Gulf Breeze, FL 32561}}
	\\{\textit {\normalsize Phone: 850-934-2455, Fax: 850-934-2401, Email: \href{mailto:hagy.jim@epa.gov}{hagy.jim@epa.gov}}}
  \\\\{\textit {\normalsize $^4$University of West Florida}}
	\\{\textit {\normalsize Center for Diagnostics and Bioremediation, 11000 University Parkway, Pensacola, FL 32514}}
	\\{\textit {\normalsize Phone: 850-857-6089, Fax: 850-474-3130, Email: \href{mailto:jcaffrey@uwf.edu}{jcaffrey@uwf.edu}}}
	}
\date{}
\maketitle
\vfill{\centerline{\textit {\normalsize Running head: Noise in Estuary Metabolism}}}
\end{singlespace}
\clearpage

%%%%%%
% acknowledgments
\section{Acknowledgments}

We acknowledge the significant efforts of research staff and field crews from the \acl{SWMP} of the \acl{NERRS} for providing access to high quality data sets.  We thank Dr. Jane Caffrey for stimulating discussion and previous work on applications of the open-water method to estuarine monitoring data. This study was funded by the US \acl{EPA}, but the contents are solely the views of the authors.  Use of trade names does not constitute endorsement by the US government.

%%%%%%
% abstract
\newpage
\section{Abstract}
Time series of \ac{DO} can be used to estimate rates of primary production and respiration in aquatic ecosystems.  However, continuous monitoring data at estuaries may reflect variation from both biological and physical processes, such that observed data may produce inaccurate or misleading estimates of metabolism.  Statistical techniques that dynamically quantify variation in \ac{DO} and tidal changes over time have the potential to isolate biological signals in \ac{DO} variation to more accurately estimate metabolism.  A weighted regression method was developed to normalize, or detide, the predicted \ac{DO} signal to remove the influence of physical advection on metabolism estimates.  The method was tested using a simulation approach to create multiple \ac{DO} time series with known additive components of biological and physical variation on different periods.  The method was further validated using one year of continuous monitoring data from four case studies. We provide a detailed discussion on use of the method for improving certainty in evaluation of \ac{DO} measurements from sites with strong tidal influences.  Moreover, we propose that the method could expand use of the open-water method for estimating ecosystem metabolism in estuaries given that the approach can produce robust estimates of \ac{DO} that are independent of tidal advection.  In particular, this could facilitate the use of shorter deployment periods for water quality monitors or incomplete time series given that known biases related to water movement could be removed. 

\acresetall
\clearpage

%%%%%%
% intro
\section{Introduction} \label{intro}

Time series of dissolved oxygen are increasingly used to estimate ecosystem metabolism \citep{Kemp12,Needoba12}.  Integrated measures of metabolism describe the balance between production and respiration processes that create and consume organic matter, respectively.  Although metabolic rates vary naturally at different spatial and temporal scales \citep{Ziegler98,Caffrey04,Russell07}, anthropogenic nutrient sources are often contributing factors that increase rates of production (\citealt{Nixon95}, \citetalias{NRC00}).  Inputs of limiting nutrients beyond background concentrations may decrease the resilience of an ecosystem such that higher rates of production are coupled with higher biological oxygen demand \citep{Yin04,Kemp09}.  Cultural eutrophication is frequently linked to declines in water quality through lower levels of dissolved oxygen, degradation in aquatic vegetation habitat, and increased frequency of harmful algal blooms \citep{Cloern96,Short96,Rabalais02,Diaz08}.  Reliables estimates of ecosystem metabolism are critical for measuring both background rates of production and potential impacts of human activities on ecosytem condition.     

Open-water techniques have been used for decades to infer metabolic rates using \textit{in situ} measurements from continuous monitoring data \citep{Odum56}. Daily integrated measurements of metabolism represent the balance between daytime production and nighttime respiration.  The open-water method uses the diel fluctuation of dissolved oxygen to estimate ecosytem metabolism, after correcting for air-water gas exchange \citep{Kemp12}.  As with any method, the ability to accurately estimate whole system metabolism depends on the degree to which assumptions of the theory are met.  The fundamental assumption is that the time series of \ac{DO} describes the same water mass over time \citep{Needoba12}.  Estimates of metabolism may be inaccurate if substantial variation in water column mixing occurs throughout the period of observation \citep{Kemp80,Russell07}.  Given this critical challenge, the open-water method has been used with varying success in lakes \citep{Staehr10,Coloso11,Batt12} and estuaries \citep{Caffrey04,Russell07,Caffrey13}.  Appropriate placement of monitoring sondes, sampling frequency and duration, and reliability of data from single stations have been relevant issues in applying the open-water method to systems influenced by physical mixing \citep{Russell07,Staehr10}.  Application of the method to estuaries is a particular concern as physical mixing caused by tidal currents may confound the biological variation in \ac{DO} time series \citep{Kemp80,Caffrey03,Nidzieko14}.  Individual sampling stations near bay inlets or along major tidal axes may produce \ac{DO} time series that fail to meet the assumptions of the open-water method.   

Although numerous studies have shown that application of the open-water method to lakes or estuaries may be problematic \citep{Ziegler98,Caffrey03,Coloso11,Batt12,Nidzieko14}, very few quantitative approaches have been developed to address potential bias or noise in \ac{DO} signals from phyical advection.  For example, an extensive analysis by \citet{Caffrey03} applied the open-water method to estimate metabolism at 28 continuous monitoring stations at 14 US estuaries.  A significant portion of the production and respiration estimates were negative (3 - 69\% depending on site), suggesting significant variation from advection of water masses with different metabolic histories was a likely factor influencing the \ac{DO} time series.  These `anomalous' values are typically omitted from the analysis \citep{Caffrey03,Collins13}, which may upwardly bias estimates of metabolism \citep{Murrell13}.  Further, \citet{Nidzieko14} evaluated the effects of tidal advection on metabolism estimates in a mesotidal estuary.  Estimates from a single location were strongly correlated with the spring-neap cycle such that net heterotrophy was more common during spring tides, whereas metabolism was generally balanced during neap tides.  A control-volume approach was used by impounding a section of the upper estuary to understand how physical processes contribute to biological variability.  Although useful as an \textit{in situ}, site-specific approach, more accessible statistical methods specific to time series are needed given the increasing availability of continuous monitoring data. For example, \citet{Batt12} acknowledged this need by exploring the use of a Kalman filter \citep{Harvey89} to remove process and observation uncertainty from \ac{DO} time series in lakes.  Similar approaches have not been developed for estuaries, particularly those that address potential effects of tidal advection.

This article describes the application of a method for filtering an observed \ac{DO} time series for estimated tidal effects to more accurately quantify estimates of ecosystem metabolism for estuaries.  Specifically, the apparent effects of tidal advection on \ac{DO} observations are removed to improve the fidelity of open-water metabolism estimates derived from continuous water quality data.  We used a weighted regression approach originally developed to resolve trends in pollutant concentrations in streams and rivers \citep{Hirsch10}.  The weighted regression approach creates dynamic predictions of \ac{DO} as a function of time and tidal height change, which are then used to filter, or detide, the \ac{DO} signal.  First, we used simulated \ac{DO} time series with known characteristics to evaluate ability of the weighted regression to remove the simulated effects of a tidally-advected \ac{DO} gradient.  Second, the simulation results informed the application of the method to four case studies chosen from the \aclu{NERRS} (\acs{NERRS}, \citealt{Wenner04}).  Specifically, one year of \ac{DO} time series for each case study was filtered to adjust estimates of ecosystem metabolism to apparent tidal effects.  In all examples, tidal height is used as a proxy for lateral water movements that may influence \ac{DO} observations.  In the absence of quantitative data describing lateral \ac{DO} variation (e.g., contemporaneous stations along a tidal axis), we assume tidal height is an appropriate measure that approximates lateral variation.  Accordingly, `tidal variation' or `changes in tidal height' are used throughout in reference to assumed lateral \ac{DO} gradients that are carried past monitoring sensors by tidal currents.  Overall, the analysis is meant to better characterize the relative roles of biological and physical processes in estuarine systems.

%%%%%%
% materials and procedures
\section{Materials and Procedures}

\subsection{Weighted regression for modelling and filtering \ac{DO} time series}

For this study, we adapted a weighted regression model to filter \ac{DO} time series for apparent tidal effects.  This model relied heavily on concepts used to develop the \ac{WRTDS} method for estimating pollutant concentrations in streams and rivers \citep{Hirsch10}.  The functional form of the model is:
\begin{equation}\label{funform}
DO_{obs}= \beta_0 + \beta_1 t + \beta_2 H
\end{equation}
where $DO_{obs}$ is a linear function of time $t$ and tidal height $H$.  Time is a continuous variable for the day and time of each observation as a proportion of the number of total observations added to each day.  The beginning of each day was considered the nearest thirty minute observation to sunrise for the location.  Our model differed from the original \ac{WRTDS} method that included parameters to estimate variation of the response variable on a sinuisoidal period.  \ac{DO} variation was not modeled using this approach to avoid constraining parameter estimates by periodic, diel components.

Weighted regression was implemented as a moving window that allowed for estimation of \ac{DO} throughout the time series by adapting to variation through time as a function of tide. Regression models were estimated sequentially for each observation in the time series using dynamic weight vectors that change with the center of the window.  Weight vectors quantified the relevance of observations to the center of the window in respect to time, hour of the day, and tidal height.  Specifically, weights were assigned to each variable using a tri-cube weighting function \citep{Tukey77,Hirsch10}:
\begin{equation}
w= \left\{ 
  \begin{array}{l l}
    \left(1-\left(d/h\right)^3\right)^3 & \quad \textrm{if } |d| \leq h \\
    0 & \quad \textrm{if } |d| > h 
  \end{array} \right.
\end{equation}
where the weight $w$ of each observation is inversely proportional to the distance $d$ from the center of the window such that observations more similar to the point of reference are given higher importance in the regression.  Weights exceeding the maximum width of the window $h$ are equal to zero.  The tri-cube weighting function is similar to a Gaussian distribution such that weights decrease gradually from the center until the maximum window width is reached.  Regressions that use simpler windows (e.g., boxcar approach) are  more sensitive to influential observations as they enter or leave the window, whereas the tri-cube function minimizes their effect through gradual weighting of observations from the center \citep{Hirsch10}.  The final weight vector for each observation is the product of three separate weight vectors for time (day), hour, and tidal height. Windows for time and hour weight observations based on distance (time) from the center of the window.  The window for tidal height weights observations based on the difference from the center as a proportion of the total tidal height range.  For example, a half-window width of 0.5 means that observations are weighted proportionately within +/- 50\% the total range referenced to the tidal height in the center of the window. A low weight is given to an observation if any of the three weighting values were not similar to the center of the window since the final weight vector is the product of three weight vectors for each variable (see the link in the \hyperref[multi]{multimedia} section for graphical display of different weights).    

The choice of window widths for weight vectors strongly affects the model results.  Excessively large or small window widths may respectively under- or over-fit the observed data.  Accordingly, appropriate window widths depend on the objective for using the model.  The weighted regression approach can be used for both predicting observed \ac{DO} and filtering the observed time series to remove the variance that coincided with the tidal cycle.  Window widths that minimize prediction error or fit to the observed data are typically smaller than widths that would be used for filtering tidal effects.  Similarly, window widths that more effectively filter the \ac{DO} signal may produce imprecise predictions for the observed data.  Evaluations of the weighted regression method with simulated \ac{DO} time series, described below, used multiple window widths to evaluate the ability of the model to filter the \ac{DO} signal.  The ability to predict observed \ac{DO} was not a primary objective such that the window widths were evaluated only in the context of removing tidal variation from the \ac{DO} time series.  

The approach to filter physical advection from the observed \ac{DO} time series differs slightly from methods in \citet{Hirsch10}.  The previous approach used a two-dimensional grid predicted for stream pollutant concentrations across the time series and the range of discharge values observed in the study system \citep{Hirsch10}.  Normalized or discharge-independent values for pollutant concentration were obtained by averaging grid predictions across the discharge values that were likely to occur on a given day.  Rather than creating a two-dimensional grid of \ac{DO} related to time and tidal height change, the normalized time series herein were the model predictions conditional on time and constant tidal height set to the mean:
\begin{equation} \label{do_nrm}
DO_{nrm} = f(DO_{obs}|\bar{H}, t)
\end{equation}
such that the normalized time series represents \ac{DO} variation related to biological processes.  The term `filter' is used in reference to the removal of a specific variance component from the time series, while maintaining the structure of the biological component.  Although the approach shares similarities with common filtering techniques, a distinction is noted such that weighted regression has a specific purpose rather than the more generic objectives of common filters (e.g., moving window averages or local smoothers, \citealt{Shumway11}).     

%%%%%%
% assessment
\section{Assessment}

\subsection{Simulation of \ac{DO} time series}

To test the ability of the weighted regression to filter the \ac{DO} signal for apparent tide effects, multiple time series with known characteristics were simulated and filtered.  A simulation approach was used prior to application with real data given that the true biological signal can be created as a known component for comparison with the filtered results from weighted regression. The following describes the theoretical basis for developing the simulated time series.  Observed \ac{DO} time series were simulated as the sum of variation from biological processes and physical effects related to tidal advection:  
\begin{equation} \label{do_obs}
DO_{obs} = DO_{bio} + DO_{adv}
\end{equation}
Biological \ac{DO} signals are inherently noisy \citep{Batt12} and variance can be further described as:
\begin{equation} \label{do_bio} 
DO_{bio} = DO_{die} + DO_{unc}
\end{equation} 
\begin{equation} \label{do_unc}
DO_{unc} = \epsilon_{obs} + \epsilon_{proc}
\end{equation}
where the biological \ac{DO} signal ($DO_{bio}$) is the sum of diel variation ($DO_{die}$) plus uncertainty or noise ($DO_{unc}$).  Total uncertainty in the biological \ac{DO} signal is described as variation from observation and process uncertainty \citep[$\epsilon_{obs}$ and $\epsilon_{pro}$,][]{Hilborn97}.  Multiple time series at 30 minute time steps over 30 days were created by varying the relative magnitudes of each of the components of observed \ac{DO} in \cref{do_obs,do_bio,do_unc} to test the effectiveness of weighted regression under different scenarios.  Accordingly, observed \ac{DO} was generalized as the additive combination of four separate time series (\cref{fig:do_sim}):
\begin{equation} \label{do_obs_all}
DO_{obs} = DO_{adv} + DO_{die} + \epsilon_{obs} + \epsilon_{pro}
\end{equation} 

Each component of the simulated time series was created as follows.  First, the diel component, $DO_{die}$, was estimated \citep{Cryer08}:
\begin{equation} \label{do_sin}
DO_{die} = \alpha + \beta\cos\left(2\pi ft + \Phi\right)
\end{equation}
such that the mean DO ($\alpha$) was 8, amplitude ($\beta$) was 1, $f$ was 1/48 to represent 30 minute intervals, $t$ was the time series vector and $\Phi$ was the x-axis origin set for an arbitrary sunrise at 630.  The diel signal was increasing during the day and decreasing during the night for each 24 hour period and ranged from 7 to 9 mg L$^{-1}$.  Uncertainty was added to the diel \ac{DO} signal as the sum of observation and process uncertainty:
\begin{equation} \label{do_unc_n}
DO_{unc, n} = \epsilon_{obs, n} + \int_{t = 1}^{n} \epsilon_{pro, t}
\end{equation}
where observation and process uncertainty ($\epsilon_{obs}$, $\epsilon_{pro}$) were simulated as normally distributed random variables with mean zero and standard deviation varying from zero to an upper limit, described below.  Process uncertainty was estimated as a serially correlated variable using the cumulative sum of $n$ observations plus random variation added at each time step for $t = 1, ..., n$.  The total uncertainty, $DO_{unc}$, was added to the diel \ac{DO} time series to create the biological \ac{DO} time series (\cref{do_bio,fig:do_sim}).

A semidiurnal tidal series was simulated with a period of 12.5 hours to represent the principal lunar component \citep{Foreman89}.  The amplitude was set to 1 meter and centered  at 4 meters.  The tidal time series simulated \ac{DO} changes with advection, $DO_{adv}$ (\cref{do_obs_all,fig:do_sim}). Conceptually, this vector represents the rate of change in \ac{DO} as a function of horizontal water movement from tidal advection such that:
\begin{equation} \label{deltdo}
\frac{\delta DO_{adv}}{\delta t} = \frac{\delta DO}{\delta x} \cdot \frac{\delta x}{\delta t}
\end{equation}
\begin{equation} \label{deltx}
\frac{\delta x}{\delta t} = k \cdot \frac{\delta H}{\delta t}
\end{equation}
where the first derivative of the tidal time series, as change in height over time $\delta H / \delta t$, is multiplied by a constant $k$, to estimate horizontal tidal excursion over time, $\delta x / \delta t$.  The horizontal excursion is assumed to be associated with a horizontal \ac{DO} change, $\delta DO / \delta x$, such that the product of the two estimates the \ac{DO} change at each time step from advection, $DO_{adv}$. In practice, the simulated tidal signal was used to estimate $DO_{adv}$:
\begin{equation} \label{do_advp}
DO_{adv} \propto H
\end{equation}
\begin{equation} \label{do_adv}
DO_{adv} = 2\cdot a + a \cdot \frac{H- \min H}{\max H - \min H}
\end{equation}
where $a$ is analogous to $k$ in \cref{deltx} and is chosen as the transformation parameter to standardize change in \ac{DO} from tidal height change to desired units.  For example, $a = 1$ will convert $H$ to a scale that simulates changes in \ac{DO} from tidal advection that range from +/- 1 mg L$^{-1}$.  The final time series for observed \ac{DO} was the sum of biological \ac{DO} and advection \ac{DO} (\cref{do_obs,fig:do_sim}).

\subsection{Evaluation of weighted regression with simulated \ac{DO} time series}

Multiple time series were simulated by varying the conditions in \cref{do_obs_all} ((\cref{fig:sim_ex})) to evaluate weighted regression under difference conditions. Specifically, the simulated data varied in the relative amount of noise in the measurement ($e_{pro}$, $e_{obs}$), relative amplitude of the diel \ac{DO} component ($DO_{die}$), and degree of association of the tide with the \ac{DO} signal ($DO_{adv}$).  Three levels were evaluated for each variable: relative noise as 0, 1, and 2 standard deviations for both process and observation uncertainty, amplitude of diel biological \ac{DO} as 0, 1, and 2 mg L$^{-1}$, and \ac{DO} change from tidal advection as 0, 1, and 2 mg L$^{-1}$. A total of 81 time series were created based on the unique combinations of parameters (\cref{fig:sim_ex}).  Half-window widths (day, hour of day, and tide height) for the weighted regressions were evaluated for each time series: time as 1, 3, and 6 days, time of day as 1, 3, and 6 hours, and tidal height as 0.25, 0.5, and 1 as a proportion of the total range given the height at the center of the window.  The window widths were chosen based on preliminary assessments that suggested a large range in model performance was described by these values.  In total, 27 window width combinations were evaluated for each of 81 simulated time series, producing results for 2187 weighted regressions.

The filtered \ac{DO} time series were compared to the simulated data to evaluate the ability of weighted regression to characterize the biological \ac{DO} time series in \cref{do_obs}. Comparisons were made using Pearson correlation coefficients and the \ac{RMSE}.  Overall, the weighted regressions produced filtered time series that were similar to the `true' biological time series regardless of the simulation parameters (\cref{tab:dtd_perf1}) or window widths (\cref{tab:dtd_perf2}, results for each simulation can be viewed using the link in the \hyperref[multi]{multimedia} section).  The median correlation between the filtered and biological values for all time series and window widths was 0.59, with values ranging from -0.78 (very poor) to 1.00 (perfect).  Mean error was 1.10, with values ranging from 0 (perfect) to 2.40 (very poor).  Simulations with very poor performance were those that had minimum widths for day windows and maximum widths for hour windows, or were those with the \ac{DO} signal composed entirely of noise from observation uncertainty. As expected, simulations with no biological or tidal influence had filtered time series that were identical to the true time series (e.g., correlation of one, \ac{RMSE} of zero).  

Characteristics of \ac{DO} time series that contributed to improved model performance were increasing amplitude of the diel \ac{DO} component ($DO_{die}$) and increasing process error ($e_{pro}$), whereas increasing observation error contributed to decreased performance (\cref{tab:dtd_perf1,fig:err_surf1}).  Model performance decreased slightly with increasing tidal effects (i.e., increasing magnitude of $DO_{adv}$).  Increasing widths for day and tidal height windows contributed to improved model performance, whereas reduced performance was observed with increasing hour windows (\cref{tab:dtd_perf2,fig:err_surf2}).  Graphical summaries of model performance by simulation parameters (\cref{fig:err_surf1}) and half window widths (\cref{fig:err_surf2}) support the general trends described by \cref{tab:dtd_perf1,tab:dtd_perf2}.

\subsection{Validation of weighted regression with case studies}

Results from the simulated time series were used to inform the validation of weighted regression with real data, specifically with respect to choosing half-window widths described below.  Continuous monitoring data from the \acl{NERRS} was used to validate the weighted regression model by evaluating estimates of ecosytem metabolism obtained from observed and filtered \ac{DO} time series. \ac{NERRS} is a federally-funded network of 28 protected estuaries established for long-term research, water-quality monitoring, education, and coastal stewardship \citep{Wenner04}.  Continuous water quality data have been collected at \ac{NERRS} sites since 1994 through the \aclu{SWMP} (\acs{SWMP}, \citetalias{CDMO14}).  In addition to providing a basis for trend evaluation, data from \ac{SWMP} provides an ideal opportunity to evaluate long-term variation in water quality parameters from biological and physical processes.  Continuous \ac{SWMP} data can be used to describe \ac{DO} variation at sites with different characteristics, including variation from ranges in tidal regime \citep{Sanger02} and rates of ecosystem production \citep{Caffrey03,Caffrey04}.  We selected sites from the \ac{SWMP} database that had desirable characteristics for validating weighted regression.  Specifically, four macrotidal sites were chosen based on apparent relationships between \ac{DO} and tidal changes (\cref{fig:case_map,tab:case_att}): Vierra Mouth station at Elkhorn Slough (California, 36.81$^{\circ}$N, 121.78$^{\circ}$W), Bayview Channel at Padilla Bay (Washington, 48.50$^{\circ}$N 122.50$^{\circ}$W), Middle Blackwater River station at Rookery Bay (Florida, 25.93$^{\circ}$N 81.60$^{\circ}$W), and Dean Creek station at Sapelo Island (Georgia, 31.39$^{\circ}$N 81.28$^{\circ}$W).   

The weighted regression model was applied to continuous \ac{DO} time series and water level measurements from January 1\textsuperscript{st} to December 31\textsuperscript{st} 2012 at the four sites.  Tide predictions were obtained for each site using harmonic regression applied to the sonde depth data (\texttt{oce} package in R, \citealt{Foreman89}, \citetalias{RDCT14}). The stations were generally semidiurnal or mixed semidiurnal and net heterotrophic on an annual basis (\cref{tab:case_att}).  Net heterotrophy (i.e., respiration exceeding production) is typical for shallow water systems at temperate latitudes \citep{Caffrey03}, although values in \cref{tab:case_att} were from observed \ac{DO} time series that were strongly correlated with water level height.

\subsection{Estimates of ecosystem metabolism before and after filtering} \label{met_sec}

The weighted regression method was applied to the annual data for each station to obtain a filtered \ac{DO} time series for estimating metabolism.  Ecosystem metabolism was estimated using the open-water technique \citep{Odum56} as described in \citet{Caffrey13}.  The method is used to infer net ecosystem metabolism using the mass balance equation:
\begin{equation} \label{metrate}
\frac{\delta DO}{\delta t} = P - R + D
\end{equation}
where the change in \ac{DO} concentration ($\delta DO$, g O$_2$ m$^{-3}$) over time ($\delta t$, hours) is equal to photosynthetic rate ($P$, g O$_2$ m$^{-3}$ hr$^{-1}$) minus respiration rate ($R$, g O$_2$ m$^{-3}$ hr$^{-1}$), corrected for air-sea gas exchange ($D$, g O$_2$ m$^{-3}$ hr$^{-1}$) \citep{Caffrey13}. $D$ is estimated as the difference between the \ac{DO} saturation concentration and observed \ac{DO} concentration, multiplied by a volumetric reaeration coefficient, $k_a$ \citep{Thebault08}.  The diffusion-corrected \ac{DO} flux estimates were averaged during day and night for each 24 hour period in the time series, where flux is an hourly rate of \ac{DO} change.  Respiration rates were assumed constant during the night and substracted from daily net production estimates to yield gross production (\cref{tab:case_att}).  

Half window widths of six days, one hour, and a tidal proportion of one half were used to filter the observed \ac{DO} time series.  Although the selection of window widths involves a degree of subjectivity, results from the simulations suggested that these values were appropriate for filtering \ac{DO} times series within the constraints of the analysis. Unlike the simulated data, the true biological \ac{DO} signal was unknown for the case studies.  Accordingly, the regression results were evaluated  using correlations of \ac{DO} and metabolism estimates with tidal height before and after application of the model.  Daily metabolism estimates before and after filtering were compared to the mean rate of tidal height change (i.e., first derivative of the predicted tidal height) for each day during separate solar periods.  Production rates were compared to mean rates of tidal height change during the day, respiration rates were compared to mean rates of change during the night, and net metabolism rates were compared to mean rates of change for the total 24 hour period each day.  Results were also evaluated based on the occurrence of `anomalous' daily production or respiration estimates, where anomalous was defined as negative production during the day and positive respiration estimates during the night.  Anomalous values have been previously attributed to the effects of physical processes on \ac{DO} time series \citep{Caffrey03}. Although anomalies could be caused by processes other than tidal advection, e.g., abiotic dark oxygen production \citep{Pamatmat97}, we assumed that physical processes were the dominant sources of these values given the tidal characteristics at each site.  Finally, means and standard errors of metabolism estimates were evaluated before and after filtering to determine if annual aggregations were significantly different.   

Filtering had significant effects on the correlations between water level changes, \ac{DO} time series, and daily integrated metabolism estimates (\cref{tab:cor_res}, see the link in the \hyperref[multi]{multimedia} section for graphical results of each case study).  Correlations of observed \ac{DO} time series with predicted tidal height were highly significant and positive at all sites, except Padilla Bay where increases in water level were associated with decreases in \ac{DO}  concentration.  The filtered \ac{DO} time series had greatly reduced correlations with tidal height, although relationships were still significant after filtering likely because of the large sample size for each site (n $\approx$ 17,500). Comparison of metabolic rates to tidal changes before and after filtering produced inconsistent results (\cref{tab:cor_res}).  Correlations for Elkhorn Slough and Sapelo Island showed consistent reductions in all three metabolims estimates after filtering.  Correlations for Padilla Bay and Rookery Bay were of opposite sign and greater magnitude after filtering for production and respiration, although net metabolism estimates had reduced correlations.  

The proportion of daily integrated metabolism estimates that were anomalous (negative production, positive respiration) were significantly reduced for most sites after filtering (\cref{tab:case_res}), perhaps indicating the relative effects of water movement.  Before filtering, anomalous values ranged from 0.09 (as a proportion of the total estimates, Rookery Bay) to 0.22 (Padilla Bay) for production and 0.08 (Rookery Bay) to 0.21 (Elkhorn Slough) for respiration. Anomalous values were reduced to near zero for Rookery Bay and Sapelo Island, by approximately half for Padilla Bay (0.13 for production, 0.13 for respiration), and only slightly reduced for Elkhorn Slough (0.17 for production, 0.17 for respiration).  Metabolism estimates using filtered \ac{DO} time series had decreased mean production (-55.5 \% change from the annual mean) and respiration (-55.2 \%) for Elkhorn Slough, increased mean production (74.0 \%) and respiration (74.8 \%) for Padilla Bay, and generally unchanged mean production and respiration for Rookery Bay and Sapelo Island (\cref{tab:case_res}).  Mean net ecosystem metabolism was unchanged for all sites.  Decreases in the standard erorr for all metabolism estimates (production, respiration, and net) were observed for all cases after filtering.  

An example from Sapelo Island illustrates the effects of weighted regression on \ac{DO} and metabolism estimates (\cref{fig:phase_out,fig:phase_in,fig:case_ex}).   A two-week period in February showed when the tidal cycles were both in and out of phase with the diel cycling, where phasing describes synchronicity between maximum tide heights and day/night periods \citep{Nidzieko14}.  That is, maximum tide heights were generally out of phase with the diel cycle during the first week when low tides were observed during the middle of the night and the middle of the day (\cref{fig:phase_out}), whereas tide heights were in phase during the second week when the maximum tide height occured during the day and night (\cref{fig:phase_in}).  The effects of tidal height change on the observed \ac{DO} time series are visually apparent in the plots. The first week illustrates a strong negative bias (less respiration, less production) in the observed \ac{DO} signal from low tides at mid-day and mid-night, whereas the second example illustrates a strong positive bias (more respiration, more production) in the observed \ac{DO} from high tides. These biases are apparent in the metabolism estimates using the observed data (\cref{fig:case_ex}).  Anomalous estimates occur when low tides are in phase with the solar cycle (week one), whereas metabolism estimates are likely over-estimated when high tides are in phase with the solar cycle (week two).  The filtered time series shows noticeable changes given the direction of bias from the phasing between tidal height and diel period.  \ac{DO} values were higher after filtering when low tides occurred during night and day periods, whereas \ac{DO} values were lower after filtering when high tides occurred during day and night periods (\cref{fig:phase_out,fig:phase_in}).  Changes in metabolism estimates after filtering were also apparent, such that the anomalous values were removed during the first week and the positive bias in the second week is decreased (\cref{fig:case_ex}).

\subsection{Effects of aggregation and importance of filtering}

A point of concern is the period of observation within which observed \ac{DO} is affected by tidal height changes and the extent to which this affects the interpretation of ecosystem metabolism.  The effects of tidal variation on daily estimates may not be relevant if seasonal or annual aggregations remove this potential bias.  The example from Sapelo Island in the previous section highlights this point given that mean production and respiration estimates before and after filtering were generally unchanged for the two-week period. \cref{tab:case_res} also indicated that mean annual estimates of production and respiration were unchanged for Rookery Bay and Sapelo Island.  However, annual averages of production and respiration estimates were significantly different for Elkhorn Slough and Padilla Bay. Given these results, tidal variation may or may not have effects on metabolism estimates on aggregated time scales longer than 24 hours, depending on the location.  Therefore, an evaluation of weighted regression to filter the effects of tidal variation on ecosystem metabolism for different periods of observation is critical for its application.  Specifically, when should filtering be applied if aggregation of observed data on longer time periods removes potential bias?  A comparison of observed and filtered estimates that are aggregated over different periods of observation (e.g., annual, seasonal, monthly) could help address this question.

The observed and filtered daily estimates were averaged by month and season (Fall, Spring, Summer, and Winter) for each case study to evaluate effects of aggregation on mean production and respiration.  Mean annual estimates in \cref{tab:case_res} also provided a basis of comparison with monthly and seasonal aggregation. Significant variation in aggregated production and respiration estimates for month and season was observed for each case study (\cref{fig:metab_sum1,fig:metab_sum2}).  Filtered production and respiration estimates for Padilla Bay and Rookery Bay exhibited seasonal and monthly variation that was more characteristic of expected trends during warmer months.  Specifically, production estimates based on observed \ac{DO} were substantially muted for both Padilla Bay (\cref{fig:metab_sum1}) and Rookery Bay (\cref{fig:metab_sum2}) during summer months, whereas values were significantly higher after filtering. Results for Sapelo Island suggested that winter and summer months were under- and over-estimated, respectively, based on the observed data.  Results for Elkhorn Slough varied significantly such that production and respiration were significantly reduced after filtering regardless of the aggregation period.  Overall, these trends emphasize the importance of considering different aggregation periods for interpreting metabolism estimates.  Each case study showed differences in observed and filtered values at monthly and seasonal aggregations, whereas only two of the four case studies had mean aggregated estimates that were substantially different (Elkhorn Slough and Padilla Bay, \cref{tab:case_res}).  Periods of observation as long as one year may include significant sources of bias from tidal advection, suggesting the need for applying weighted regression given careful consideration of appropriate window widths.       

 

%%%%%%
% discussion
\section{Discussion}

The weighted regression approach was developed to improve estimates of ecosystem metabolism by removing variation associated with tidal change in observed \ac{DO} time series.  The application to simulated \ac{DO} time series with known characteristics and extension to continuous monitoring data from selected \ac{NERRS} sites suggested the approach can isolate and remove variation in observed \ac{DO} from tidal change.  Further, aggregation of metabolism estimates using the filtered \ac{DO} time series were significantly different than those using the observed data, particularly for relatively long periods of observation depending on location. These results suggest that previous estimates of annual means may not accurately reflect true metabolic signals if the effects of tidal variation confound biological signals in observed \ac{DO} time series.  Additionally, variation of aggregated metabolism estimates were substantially reduced after filtering, suggesting greater confidence in interpreting estimates even if the mean values are similar.

Comparisons between filtered and biological \ac{DO} time series from the simulations indicated that weighted regression can reduce the effects of tidal variation for a range of characteristics of \ac{DO} time series.  An examination of scenarios that produced abnormal results can provide additional insight into factors that affect the performance of weighted regression.  For example, poor performance was observed when the observation uncertainty ($\epsilon_{obs}$) was high and both process uncertainty ($\epsilon_{pro}$) and tidal advection ($DO_{adv}$) were low.  These examples represent time series with excessive random variation, no auto-correlation, and no tidal influence.  Poor performance is expected because the weighted regression models a non-existent tidal signal in a very noisy \ac{DO} time series.  These results were observed even for time series with a large diel component of the biological \ac{DO} signal, suggesting that the model will produce random results in microtidal systems with high noise and no serial correlation.  From a practical perspective, weighted regression should not be applied to noisy time series if there is not sufficient evidence to suggest the variation is related to tidal changes.  Alternative approaches, such as the Kalman filter \citep{Harvey89,Batt12}, may be more appropriate if random variation is the primary source of uncertainty.  Similarly, results with perfect or near-perfect correlations between filtered and biological \ac{DO} time series were observed when observation uncertainty and tidal effects were not components of the simulated time series.  Although there is no need to apply weighted regression to time series with no apparent tidal influences, the results will not be incorrect.  We emphasize that the weighted regression should only be applied to time series for which specific conditions apply, as described in the recommendations below.  

Correlations of metabolism estimates with tidal height changes after filtering were generally reduced, although trends were not always consistent.  However, correlations of net metabolism estimates were reduced in all cases.  An additional indication of the effectivenes of weighted regression was the reduction of anomalous metabolism estimates after filtering for all case studies.  Negative production and positive respiration estimates suggest assumptions of the open-water method are violated \citep{Needoba12}, although `normal' estimates (positive production and negative respiration) may still include a significant source of bias from physical advection by providing over-estimates of true values.  For example, \citet{Nidzieko14} observed that net metabolism at Elkhorn Slough was strongly heterotrophic during spring tides that occurred at nighttime such that inundation of salt marshes during the night following by draining with low tide during the day lead to inflated respiration values.   Synchrony between solar and tidal cycles is a critical concern for interpreting metabolism estimates, although a broader discussion regarding whether or not this represents an actual bias in metabolism from physical advection may be needed. 

The weighted regression approach makes no assumptions as to the relationships between \ac{DO} and tidal variation over time.  Although the functional form of the model is a simple linear regression with only two explanatory variables (\cref{funform}), the moving window approach combined with the adaptive weighting scheme allows for quantification of complex tidal effects that may not be possible using alternative approaches.  A similer approach by \citet{Batt12} uses a Kalman filter to improve estimates of ecosystem metabolism in lakes.  The approach minimizes uncertainty in observed \ac{DO} using a filter that combines information about the data generation process and the manner in which the data are observed \citep{Harvey89}.  Although a similar approach could be used for estuaries, it may not be effective given that the effects of tidal advection are not related to process or observation uncertainty.  Additionally, results from the case studies illustrated the ability of the weighted regression approach to model changes over time in the relationships between tidal change and \ac{DO}.  Results for Padilla Bay and Rookery Bay suggested that filtering had the largest effect during the summer, whereas the results for cooler months were not significantly different from the observed.  The weighted regression method produced filtered time series that accommodated seasonal variation in \ac{DO} conditional on tidal height change, whereas moving window filters or standard regression techniques would likely not have characterized these dynamic relationships.

%%%%%%
% comments and recs
\section{Comments and recommendations}

Results from the simulations and case studies suggested that weighted regression can be a practical approach for filtering \ac{DO} time series to remove the effects of physical advection on estimates of ecosystem metabolism.  However, application of the method may only be appropriate under specific situations.  The case studies were chosen based on the relatively high proportion of metabolism estimates that were anomalous and the strength of correlation between the observed \ac{DO} time series and tidal height.  Despite these similarites among the case studies, filtering had variable effects on metabolism estimates.  The results for Elkhorn Slough and Padilla Bay are of particular concern given that mean annual estimates were substantially different compared to those from the observed \ac{DO} time series.  Although the correlation of \ac{DO} and tidal height was reduced for both cases, in addition to a reduction of anomalous estimates, the relative change in mean metabolism before and after filtering suggests a more careful evaluation of the method is needed.  In particular, alternative window widths should be evaluated for the ability to remove tidal effects while preserving the biological signal.  The window widths in the above analysis may have removed variation in the \ac{DO} signal from both of these sources.    

Although the above analyses suggest the approach has merit, the case studies emphasize a critical challenge in applying weighted regression to monitoring data. Specifically, the true biological signal is not known and the relative contribution of horizontal advection to bias is not accurately quantified with the available data.  Comparative analyses between systems with varying tidal influence or within-system evaluations of multiple sites at fixed distances are necessary to further validate the performance of weighted regression.  In the absence of additional validation, we propose a precautionary approach for application of the weighted regression to monitoring data.  Weighted regression may be most effective at macrotidal sites with strong evidence of the effects of tidal advection on biological signals.  A weight-of-evidence approach should be used such that the occurrence of anomalous metabolism estimates, strong correlations between observed \ac{DO} and tide height, and clear visual patterns of tide change on \ac{DO} would suggest filtering is appropriate.  The choice of window widths may also produce varying results.  Window widths that produce large changes in mean annual estimates should be interpreted with caution.  In general, a pragmatic approach is emphasized such that results should be evaluated based on the preservation of diel variation from production while exhibiting minimal changes with the tide.  Such an approach, combined with further validation, will support informed management decisions through more accurate estimates of ecosystem metabolism.  

%%%%%%
% refs
\clearpage
\begin{singlespace}
\bibliographystyle{apalike_mine}
\bibliography{ref_diss}
\end{singlespace}
\clearpage

%%%%%%
% figures

\section{Figures}

% example of creating simulated time series
\centering\vspace*{\fill}
\begin{knitrout}
\definecolor{shadecolor}{rgb}{0.969, 0.969, 0.969}\color{fgcolor}\begin{figure}[!ht]


{\centering \includegraphics[width=\maxwidth]{figure/do_sim} 

}

\caption[Example of each component of a simulated \ac{DO} time series for testing weighted regression]{Example of each component of a simulated \ac{DO} time series for testing weighted regression.  The time series were created using \cref{do_obs,do_bio,do_unc,do_obs_all,do_sin,do_unc_n,deltdo,deltx,do_advp,do_adv}. Yellow indicates a twelve hour daylight period beginning at 630 each day.\label{fig:do_sim}}
\end{figure}


\end{knitrout}
\vfill
\clearpage

% plot of representative time series for simulation
\centering\vspace*{\fill}
\begin{knitrout}
\definecolor{shadecolor}{rgb}{0.969, 0.969, 0.969}\color{fgcolor}\begin{figure}[!ht]


{\centering \includegraphics[width=\textwidth]{figure/sim_ex} 

}

\caption[Representative examples of simulated time series of observed \ac{DO} ($DO_{obs}$, blue lines) and biological \ac{DO} ($DO_{bio}$, as a component of observed, red lines) created by varying each of four parameters]{Representative examples of simulated time series of observed \ac{DO} ($DO_{obs}$, blue lines) and biological \ac{DO} ($DO_{bio}$, as a component of observed, red lines) created by varying each of four parameters: strength of tidal association with \ac{DO} signal ($DO_{adv}$), amount of process uncertainty ($\epsilon_{pro}$), amount of observation uncertainty ($\epsilon_{obs}$), and strength of diel \ac{DO} component ($DO_{die}$).  Parameter values represent the minimum and maximum used in the simulations as mg L$^{-1}$ of \ac{DO}.\label{fig:sim_ex}}
\end{figure}


\end{knitrout}
\vfill
\clearpage

% example of error surfaces 
\centering\vspace*{\fill}
\begin{knitrout}
\definecolor{shadecolor}{rgb}{0.969, 0.969, 0.969}\color{fgcolor}\begin{figure}[!ht]


{\centering \includegraphics[width=\maxwidth]{figure/err_surf1} 

}

\caption[Heat maps of correlations and errors (\ac{RMSE}) for filtered \ac{DO} time series ($DO_{dtd}$) from weighted regression with `true' biological \ac{DO} ($DO_{bio}$) for varying simulation parameters]{Heat maps of correlations and errors (\ac{RMSE}) for filtered \ac{DO} time series ($DO_{dtd}$) from weighted regression with `true' biological \ac{DO} ($DO_{bio}$) for varying simulation parameters: strength of tidal association with \ac{DO} signal ($DO_{adv}$), amount of process uncertainty ($\epsilon_{pro}$), amount of observation observation uncertainty ($\epsilon_{obs}$), and strength of diel \ac{DO} component ($DO_{die}$).  Each tile represents the correlation or error from results for a given combination of simulation parameters averaged for all window widths (\cref{fig:err_surf2}).\label{fig:err_surf1}}
\end{figure}


\end{knitrout}
\vfill
\clearpage

% example of error surfaces 
\centering\vspace*{\fill}
\begin{knitrout}
\definecolor{shadecolor}{rgb}{0.969, 0.969, 0.969}\color{fgcolor}\begin{figure}[!ht]


{\centering \includegraphics[width=\maxwidth]{figure/err_surf2} 

}

\caption[Heat maps of correlations and errors (\ac{RMSE}) for filtered \ac{DO} time series ($DO_{dtd}$) from weighted regression with `true' biological \ac{DO} ($DO_{bio}$) for varying half window widths]{Heat maps of correlations and errors (\ac{RMSE}) for filtered \ac{DO} time series ($DO_{dtd}$) from weighted regression with `true' biological \ac{DO} ($DO_{bio}$) for varying half window widths: days, hour of day, and proportion of tidal range.  Each tile represents the correlation or error from results for a given combination of window widths averaged for all simulation parameters (\cref{fig:err_surf1}).\label{fig:err_surf2}}
\end{figure}


\end{knitrout}
\vfill
\clearpage

% maps of each case
\centering\vspace*{\fill}
\begin{knitrout}
\definecolor{shadecolor}{rgb}{0.969, 0.969, 0.969}\color{fgcolor}\begin{figure}[!ht]


{\centering \includegraphics[width=\maxwidth]{figure/case_map} 

}

\caption[Locations of \ac{NERRS} sites used as case studies to validate weighted regression]{Locations of \ac{NERRS} sites used as case studies to validate weighted regression.  Stations at each reserve are ELKVM (Vierra Mouth at Elkhorn Slough), PDBBY (Bayview Channel at Padilla Bay), RKBMB (Middle Blackwater River at Rookery Bay), and SAPDC (Dean Creek at Sapelo Island).\label{fig:case_map}}
\end{figure}


\end{knitrout}
\vfill
\clearpage

% example from SAPHD, phase out
\centering\vspace*{\fill}
\begin{knitrout}
\definecolor{shadecolor}{rgb}{0.969, 0.969, 0.969}\color{fgcolor}\begin{figure}[!ht]


{\centering \includegraphics[width=0.8\textwidth]{figure/phase_out} 

}

\caption[Continuous \ac{DO} time series before (observed) and after (filtered) filtering with weighted regression (top) and tidal height (m) colored by total photosynthetically active radiation (bottom, mmol m$^{-2}$)]{Continuous \ac{DO} time series before (observed) and after (filtered) filtering with weighted regression (top) and tidal height (m) colored by total photosynthetically active radiation (bottom, mmol m$^{-2}$). Results are for the Sapelo Island station for a seven day period when high tide events were out of phase with diel periods, creating lower than expected observed \ac{DO} during night and day periods. Filtered values are based on a weighted regression with half window widths of six days, one hour within each day, and tidal height proportion of one half.\label{fig:phase_out}}
\end{figure}


\end{knitrout}
\vfill
\clearpage

% example from SAPHD, phase in
\centering\vspace*{\fill}
\begin{knitrout}
\definecolor{shadecolor}{rgb}{0.969, 0.969, 0.969}\color{fgcolor}\begin{figure}[!ht]


{\centering \includegraphics[width=0.8\textwidth]{figure/phase_in} 

}

\caption[Continuous \ac{DO} time series before (observed) and after (filtered) filtering with weighted regression (top) and tidal height (m) colored by total photosynthetically active radiation (bottom, mmol m$^{-2}$)]{Continuous \ac{DO} time series before (observed) and after (filtered) filtering with weighted regression (top) and tidal height (m) colored by total photosynthetically active radiation (bottom, mmol m$^{-2}$). Results are for the Sapelo Island station for a seven day period when high tide events were in phase with diel periods, creating higher than expected observed \ac{DO} during night and day periods. Filtered values are based on a weighted regression with half window widths of six days, one hour within each day, and tidal height proportion of one half.\label{fig:phase_in}}
\end{figure}


\end{knitrout}
\vfill
\clearpage

% example from SAPDC
\centering\vspace*{\fill}
\begin{knitrout}
\definecolor{shadecolor}{rgb}{0.969, 0.969, 0.969}\color{fgcolor}\begin{figure}[!ht]


{\centering \includegraphics[width=0.8\textwidth]{figure/case_ex} 

}

\caption[Example of daily mean metabolism (net ecosystem metabolism, gross production, and total respiration) before (observed) and after (filtered) filtering with weighted regression]{Example of daily mean metabolism (net ecosystem metabolism, gross production, and total respiration) before (observed) and after (filtered) filtering with weighted regression. Results are for the Sapelo Island station for a two week period in February, 2012 when high tide was out of phase with the diel cycle during the first week (\cref{fig:phase_out}) and in phase during the second week (\cref{fig:phase_in}).\label{fig:case_ex}}
\end{figure}


\end{knitrout}

% plots of summarized metabolism estimates, before/after detiding
% ELKVM, PDBBY
\centering\vspace*{\fill}
\begin{knitrout}
\definecolor{shadecolor}{rgb}{0.969, 0.969, 0.969}\color{fgcolor}\begin{figure}[!ht]


{\centering \includegraphics[width=\maxwidth]{figure/metab_sum1} 

}

\caption[Means and standard errors of daily metabolism estimates (gross production, total respiration) aggregated by month and season]{Means and standard errors of daily metabolism estimates (gross production, total respiration) aggregated by month and season.  Aggregated estimates are for Elkhorn Slough and Padilla Bay from observed and filtered \ac{DO} time series.\label{fig:metab_sum1}}
\end{figure}


\end{knitrout}
\vfill
\clearpage

% RKBMB, SAPDC
% plots of summarized metabolism estimates, before/after detiding
\centering\vspace*{\fill}
\begin{knitrout}
\definecolor{shadecolor}{rgb}{0.969, 0.969, 0.969}\color{fgcolor}\begin{figure}[!ht]


{\centering \includegraphics[width=\maxwidth]{figure/metab_sum2} 

}

\caption[Means and standard errors of daily metabolism estimates (gross production, total respiration) aggregated by month and season]{Means and standard errors of daily metabolism estimates (gross production, total respiration) aggregated by month and season.  Aggregated estimates are for Rookery Bay and Sapelo Island from observed and filtered \ac{DO} time series.  May was removed from Rookery Bay because of incomplete data.\label{fig:metab_sum2}}
\end{figure}


\end{knitrout}
\vfill
\clearpage

%%%%%%
% tables

\section{Tables}

% summary of simulation performance for detided and biological, sim parameters
% latex.default(tab, file = "", where = "h", rowlabel = "Parameter",      caption = cap.val, caption.loc = "top", rgroup = Parms, n.rgroup = rep(3,          4), cgroup = c("Correlation", "RMSE"), n.cgroup = c(5,          5), rowname = rows, colheads = rep(c("Min", "25\\textsuperscript{th}",          "Median", "75\\textsuperscript{th}", "Max"), 2), label = "tab:dtd_perf1") 
%
\begin{table}[h]
\caption{Summary (range, median, quartiles) of correlations and error estimates comparing filtered and biological \ac{DO} time series for different simulation parameters ($DO_{die}$, $DO_{adv}$, $\epsilon_{pro}$, $\epsilon_{obs}$).  Values represent averages from multiple simulations with common parameters (e.g., row one is a summary of all simulations for which the diel \ac{DO} component was zero).\label{tab:dtd_perf1}} 
\begin{center}
\begin{tabular}{llllllclllll}
\hline\hline
\multicolumn{1}{l}{\bfseries Parameter}&\multicolumn{5}{c}{\bfseries Correlation}&\multicolumn{1}{c}{\bfseries }&\multicolumn{5}{c}{\bfseries RMSE}\tabularnewline
\cline{2-6} \cline{8-12}
\multicolumn{1}{l}{}&\multicolumn{1}{c}{Min}&\multicolumn{1}{c}{25\textsuperscript{th}}&\multicolumn{1}{c}{Median}&\multicolumn{1}{c}{75\textsuperscript{th}}&\multicolumn{1}{c}{Max}&\multicolumn{1}{c}{}&\multicolumn{1}{c}{Min}&\multicolumn{1}{c}{25\textsuperscript{th}}&\multicolumn{1}{c}{Median}&\multicolumn{1}{c}{75\textsuperscript{th}}&\multicolumn{1}{c}{Max}\tabularnewline
\hline
{\bfseries $\boldsymbol{DO_{die}}$}&&&&&&&&&&&\tabularnewline
~~0&-0.78&0.30&0.51&0.82&1.00&&0.00&0.68&1.10&1.97&2.39\tabularnewline
~~1&-0.28&0.38&0.59&0.88&1.00&&0.00&0.59&1.07&1.96&2.40\tabularnewline
~~2&-0.39&0.46&0.63&0.90&1.00&&0.00&0.62&1.10&1.97&2.40\tabularnewline
\hline
{\bfseries $\boldsymbol{DO_{adv}}$}&&&&&&&&&&&\tabularnewline
~~0& 0.00&0.27&0.58&0.93&1.00&&0.00&0.34&1.00&1.96&2.12\tabularnewline
~~1&-0.78&0.37&0.58&0.83&1.00&&0.00&0.63&1.09&1.98&2.12\tabularnewline
~~2&-0.78&0.47&0.61&0.82&1.00&&0.00&0.98&1.34&1.99&2.40\tabularnewline
\hline
{\bfseries $\boldsymbol{\epsilon_{pro}}$}&&&&&&&&&&&\tabularnewline
~~0&-0.78&0.34&0.57&0.86&1.00&&0.00&0.63&1.06&1.96&2.40\tabularnewline
~~1&-0.78&0.37&0.59&0.85&1.00&&0.00&0.63&1.06&1.97&2.40\tabularnewline
~~2&-0.78&0.41&0.61&0.85&1.00&&0.00&0.63&1.11&1.98&2.40\tabularnewline
\hline
{\bfseries $\boldsymbol{\epsilon_{obs}}$}&&&&&&&&&&&\tabularnewline
~~0&-0.78&0.31&0.82&0.98&1.00&&0.00&0.29&0.76&1.50&2.40\tabularnewline
~~1& 0.05&0.37&0.58&0.81&0.99&&0.07&0.98&1.05&1.49&2.39\tabularnewline
~~2& 0.05&0.40&0.58&0.70&0.99&&0.15&1.06&1.96&2.01&2.40\tabularnewline
\hline
\end{tabular}
\end{center}
\end{table}



% summary of simulation performance for detided and biological, window widths
% latex.default(tab, file = "", rowlabel = "Window", caption = cap.val,      caption.loc = "top", rgroup = Parms, n.rgroup = rep(3, 3),      cgroup = c("Correlation", "RMSE"), n.cgroup = c(5, 5), rowname = rows,      colheads = rep(c("Min", "25\\textsuperscript{th}", "Median",          "75\\textsuperscript{th}", "Max"), 2), label = "tab:dtd_perf2") 
%
\begin{table}[!tbp]
\caption{Summary (range, median, quartiles) of correlations and error estimates comparing filtered and biological \ac{DO} time series for simulations using different half window widths in the weighted regressions (days, hours, and proportion of tidal range).  Values represent averages from multiple simulations with common window values (e.g., row one is a summary of all simulations for which the half window width was one day).\label{tab:dtd_perf2}} 
\begin{center}
\begin{tabular}{llllllclllll}
\hline\hline
\multicolumn{1}{l}{\bfseries Window}&\multicolumn{5}{c}{\bfseries Correlation}&\multicolumn{1}{c}{\bfseries }&\multicolumn{5}{c}{\bfseries RMSE}\tabularnewline
\cline{2-6} \cline{8-12}
\multicolumn{1}{l}{}&\multicolumn{1}{c}{Min}&\multicolumn{1}{c}{25\textsuperscript{th}}&\multicolumn{1}{c}{Median}&\multicolumn{1}{c}{75\textsuperscript{th}}&\multicolumn{1}{c}{Max}&\multicolumn{1}{c}{}&\multicolumn{1}{c}{Min}&\multicolumn{1}{c}{25\textsuperscript{th}}&\multicolumn{1}{c}{Median}&\multicolumn{1}{c}{75\textsuperscript{th}}&\multicolumn{1}{c}{Max}\tabularnewline
\hline
{\bfseries Days}&&&&&&&&&&&\tabularnewline
~~1&-0.78&0.63&0.89&0.97&1.00&&0.00&0.28&0.59&1.04&2.12\tabularnewline
~~3&-0.07&0.40&0.59&0.75&1.00&&0.00&0.99&1.08&1.28&2.08\tabularnewline
~~6& 0.00&0.26&0.40&0.58&1.00&&0.00&1.95&1.98&2.05&2.40\tabularnewline
\hline
{\bfseries Hours}&&&&&&&&&&&\tabularnewline
~~1&-0.78&0.36&0.58&0.82&1.00&&0.00&0.63&1.11&1.96&2.40\tabularnewline
~~3& 0.00&0.40&0.60&0.87&1.00&&0.00&0.58&1.07&1.97&2.36\tabularnewline
~~6& 0.03&0.37&0.59&0.85&1.00&&0.00&0.64&1.10&1.98&2.40\tabularnewline
\hline
{\bfseries Tide}&&&&&&&&&&&\tabularnewline
~~0.25& 0.00&0.42&0.63&0.91&1.00&&0.00&0.51&1.04&1.97&2.21\tabularnewline
~~0.5& 0.06&0.43&0.62&0.88&1.00&&0.00&0.61&1.09&1.97&2.27\tabularnewline
~~1&-0.78&0.30&0.51&0.79&1.00&&0.00&0.73&1.20&1.97&2.40\tabularnewline
\hline
\end{tabular}
\end{center}
\end{table}



% descriptive table of case studies
% latex.default(tab, file = "", rowlabel = "Site", insert.bottom = foot.val,      caption = cap.val, caption.loc = "top", cgroup = c("Tidal amplitude",          "Water quality", "Metabolism\\textsuperscript{\\textit{a}}"),      n.cgroup = c(4, 4, 3), rowname = rows, colheads = c("O1",          "P1", "M2", "S2", "DO", "Chl", "Sal", "Temp", "Pg", "Rt",          "NEM"), label = "tab:case_att") 
%
\begin{table}[!tbp]
\caption{Summary statistics of tidal component amplitudes (m), selected water quality parameters (\ac{DO} mg L$^{-1}$, chlorophyll-a $\mu$g L$^{-1}$, salinity psu, water temperature $^{\circ}$C)  and metabolism estimates (gross production, respiration, and net ecosystem metabolism as g m$^{-2}$ d$^{-1}$) for each case study.  Tidal components are principal lunar semidiurnal (O1, frequency 25.82 hours), solar diurnal (P1, 24.07 hours), lunar semidiurnal (M2, 12.42 hours), and solar semidiurnal (S2, 12 hours) estimated from harmonic regressions of tidal height (\texttt{oce} package in R, \citealt{Foreman89}, \citetalias{RDCT14}).  Water quality data are averages for the entire period of record for each site.  Metabolism estimates are means of daily integrated values.\label{tab:case_att}} 
\begin{center}
\begin{tabular}{lllllcllllclll}
\hline\hline
\multicolumn{1}{l}{\bfseries Site}&\multicolumn{4}{c}{\bfseries Tidal amplitude}&\multicolumn{1}{c}{\bfseries }&\multicolumn{4}{c}{\bfseries Water quality}&\multicolumn{1}{c}{\bfseries }&\multicolumn{3}{c}{\bfseries Metabolism\textsuperscript{\textit{a}}}\tabularnewline
\cline{2-5} \cline{7-10} \cline{12-14}
\multicolumn{1}{l}{}&\multicolumn{1}{c}{O1}&\multicolumn{1}{c}{P1}&\multicolumn{1}{c}{M2}&\multicolumn{1}{c}{S2}&\multicolumn{1}{c}{}&\multicolumn{1}{c}{DO}&\multicolumn{1}{c}{Chl}&\multicolumn{1}{c}{Sal}&\multicolumn{1}{c}{Temp}&\multicolumn{1}{c}{}&\multicolumn{1}{c}{Pg}&\multicolumn{1}{c}{Rt}&\multicolumn{1}{c}{NEM}\tabularnewline
\hline
ELKVM&0.24&0.12&0.48&0.13&&7.87&3.87&32.43&13.78&&8.14&-8.19&-0.05\tabularnewline
PDBBY&0.46&0.23&0.63&0.15&&8.97&2.24&29.17&10.44&&5.95&-5.90& 0.05\tabularnewline
RKBMB&0.13&0.04&0.36&0.10&&4.48&4.50&30.53&25.85&&3.02&-3.62&-0.60\tabularnewline
SAPDC&0.10&0.02&0.54&0.07&&4.96&5.98&27.30&21.77&&4.89&-6.04&-1.16\tabularnewline
\hline
\end{tabular}
\end{center}
\footnotesize\textsuperscript{\textit{a}}Pg: gross production, Rt: respiration, NEM: net ecosystem metabolism\end{table}



% correlations with tide before/after wtreg
% latex.default(tab, file = "", rowlabel = "Site", rgroup = unique(rows),      n.rgroup = rep(2, 4), insert.bottom = foot.val, caption = cap.val,      colheads = c("DO", "Pg\\textsuperscript{\\textit{a}}", "Rt",          "NEM"), caption.loc = "top", rowname = rep(c("Observed",          "Filtered"), 4), label = "tab:cor_res") 
%
\begin{table}[!tbp]
\caption{Correlations of tidal changes at each site with continuous \ac{DO} observations and metabolism estimates (gross production, respiration, and net metabolism) before (observed) and after (filtered) filtering with weighted regression.  \ac{DO} values are correlated with predicted tidal height at each observation, whereas metabolism estimates are correlated with mean tidal height change between observations during day, night, or total day periods for production, respiration, and net metabolism, respectively.\label{tab:cor_res}} 
\begin{center}
\begin{tabular}{lllll}
\hline\hline
\multicolumn{1}{l}{Site}&\multicolumn{1}{c}{DO}&\multicolumn{1}{c}{Pg\textsuperscript{\textit{a}}}&\multicolumn{1}{c}{Rt}&\multicolumn{1}{c}{NEM}\tabularnewline
\hline
{\bfseries ELKVM}&&&&\tabularnewline
~~Observed& 0.47***& 0.60***& 0.73***& 0.35***\tabularnewline
~~Filtered& 0.02*& 0.19***& 0.13*& 0.06 \tabularnewline
\hline
{\bfseries PDBBY}&&&&\tabularnewline
~~Observed&-0.45***&-0.33***&-0.46***&-0.25***\tabularnewline
~~Filtered& 0.07***& 0.48***& 0.47***&-0.21***\tabularnewline
\hline
{\bfseries RKBMB}&&&&\tabularnewline
~~Observed& 0.28***& 0.34***& 0.39***& 0.24***\tabularnewline
~~Filtered&-0.02**&-0.31***&-0.36***& 0.12*\tabularnewline
\hline
{\bfseries SAPDC}&&&&\tabularnewline
~~Observed& 0.48***& 0.54***& 0.71***& 0.41***\tabularnewline
~~Filtered&-0.03***& 0.16**& 0.18***&-0.05 \tabularnewline
\hline
\end{tabular}
\end{center}
\footnotesize *$p<0.05$; **$p<0.01$; ***$p<0.001$\\\textsuperscript{\textit{a}}Pg: gross production, Rt: respiration, NEM: net ecosystem metabolism\end{table}



% case study metabolism results, including perc anom
% latex.default(tab, file = "", rowlabel = "Site", insert.bottom = foot.val,      caption = cap.val, caption.loc = "top", rgroup = unique(to_tab$Site),      n.rgroup = rep(2, 4), cgroup = c("Pg\\textsuperscript{\\textit{a}}",          "Rt", "NEM"), n.cgroup = c(3, 3, 2), rowname = rows,      colheads = c(rep(c("Mean", "Std. Err.", "Anom"), 2), c("Mean",          "Std. Err.")), label = "tab:case_res") 
%
\begin{table}[!tbp]
\caption{Summary of metabolism estimates (gross production, respiration, and net metabolism) for case studies using \ac{DO} time series before (observed) and after (filtered) filtering with weighted regression.  Means and standard errors are based on daily integrated metabolism estimates.  Anomalous values are the proportion of metabolism estimates that were negative for gross production and positive for respiration.  Results are for weighted regressions with half window widths of six days, one hour within each day, and a tidal height proportion of one half.\label{tab:case_res}} 
\begin{center}
\begin{tabular}{llllclllcll}
\hline\hline
\multicolumn{1}{l}{\bfseries Site}&\multicolumn{3}{c}{\bfseries Pg\textsuperscript{\textit{a}}}&\multicolumn{1}{c}{\bfseries }&\multicolumn{3}{c}{\bfseries Rt}&\multicolumn{1}{c}{\bfseries }&\multicolumn{2}{c}{\bfseries NEM}\tabularnewline
\cline{2-4} \cline{6-8} \cline{10-11}
\multicolumn{1}{l}{}&\multicolumn{1}{c}{Mean}&\multicolumn{1}{c}{Std. Err.}&\multicolumn{1}{c}{Anom}&\multicolumn{1}{c}{}&\multicolumn{1}{c}{Mean}&\multicolumn{1}{c}{Std. Err.}&\multicolumn{1}{c}{Anom}&\multicolumn{1}{c}{}&\multicolumn{1}{c}{Mean}&\multicolumn{1}{c}{Std. Err.}\tabularnewline
\hline
{\bfseries ELKVM}&&&&&&&&&&\tabularnewline
~~Observed& 8.14&0.67&0.19&& -8.19&0.69&0.21&&-0.05&0.16\tabularnewline
~~NA& 3.63&0.23&0.17&& -3.67&0.24&0.17&&-0.04&0.05\tabularnewline
\hline
{\bfseries PDBBY}&&&&&&&&&&\tabularnewline
~~Observed& 5.95&0.69&0.22&& -5.90&0.74&0.19&& 0.05&0.22\tabularnewline
~~NA&10.36&0.63&0.13&&-10.32&0.63&0.13&& 0.04&0.08\tabularnewline
\hline
{\bfseries RKBMB}&&&&&&&&&&\tabularnewline
~~Observed& 3.02&0.14&0.09&& -3.62&0.15&0.08&&-0.60&0.06\tabularnewline
~~NA& 3.73&0.09&0.01&& -4.35&0.10&0.00&&-0.62&0.04\tabularnewline
\hline
{\bfseries SAPDC}&&&&&&&&&&\tabularnewline
~~Observed& 4.89&0.23&0.13&& -6.04&0.25&0.11&&-1.16&0.09\tabularnewline
~~NA& 4.85&0.08&0.00&& -6.04&0.10&0.00&&-1.19&0.05\tabularnewline
\hline
\end{tabular}
\end{center}
\textsuperscript{\textit{a}}Pg: gross production, Rt: respiration, NEM: net ecosystem metabolism\end{table}


\clearpage

%%%%%%
% multimedia files, appendices

\raggedbottom
\raggedright
\setlength{\parindent}{0.5in}

\section{Multimedia} \label{multi}
A simple R package with a sample dataset and code to implement weighted regression is available on GitHub, including functions to estimate ecosystem metabolism.  See the README file on the web page for download instructions and examples (\href{https://github.com/fawda123/WtRegDO}{https://github.com/fawda123/WtRegDO}).  Interactive applications are also available that illustrate the weighting scheme described in the material and procedures section (\href{https://beckmw.shinyapps.io/weights_widget}{https://beckmw.shinyapps.io/weights\_widget}), results for each simulation (\href{https://beckmw.shinyapps.io/detiding_sims/}{https://beckmw.shinyapps.io/detiding\_sims/}), and results for each case study (\href{https://beckmw.shinyapps.io/detiding_cases/}{https://beckmw.shinyapps.io/detiding\_cases/}).  Each link is a graphical summary of data based on interactive inputs to support the results in the manuscript.

\end{document}
