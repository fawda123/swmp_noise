
\begin{comment}
- Estimates of ecosystem metabolism - what it is, why we need it, basic methods

- State of the science for measuring ecosystem metabolism using open-water method - Caffrey 2004, Caffrey et al. 2013, Staehr et al. 2010, others

- problems of existing methods - Violation of assumptions or complications using Open-water method in estuaries - causes, lit review (Nidzieko et al. 2014), mostly anecdotal to date and site-specific, summary of assumptions in table 7 Staehr et al. 2010
- Collins et al. 2013 describes anomalous values (tide, near-zero detection limits), also see p. 81 in same paper, estimates of 11km related to tidal excursion, also Kemp and Boynton is cited on p. 93 in same paper as an earlier example of bio/phy early studies
- Methods for removing noise - examples
  kalhman filter
  Needoba et al. ch4, p. 91, suggest hbf filter to remove spring/neap variation, but major problem is that do variation could be affected by tides at diel periods or smaller
  averaging removes noise but what if we want to understand short-term fluctuations? 
  
- Management concerns in monitoring and ecosystem assessment --  limited resources
  
  Questions of where, frequency, and duration for deployment of sondes (see 
    Staehr et al. 2010) with limited resources
  
  Can we evaluate the extent to which monitoring provides an accurate 
    assessment of actual DO, with relevance for standards/hypoxia criteria?  
    What if we only sample during periods of increasing tide?
  
  We have x number of estuaries and y $<<$ x water quality monitors, let's put 
    them in set locations for 10 days.  What's wrong with this or what kind of 
    bias could be expected? 

- Management concerns in monitoring and ecosystem assessment -- high frequency data
  
  NERRS monitoring -- other extreme, ample data but no noise characterization
  
  Potential approaches to estimate magnitude/direction of bias from existing 
    datasets, contrast with explicit experimental design to answer same 
    question

- Study objective, goals, questions
  
  Objective -- Characterize effects of physical processes on DO signal to 
    improve estimates of metabolism with multi-year time series of high 
    frequency water quality data for US estuaries
  
  Can we describe the magnitude of the noise or bias in estimates of ecosystem 
    metabolism?
  
  Is this noise or an actual source of bias?
  
  Can stations be categorized as to the expected types of noise or bias?
  
  If so, can this noise or bias be removed?
    
    For example, what is minimum averaging window or minimum sampling time     
      required to remove noise or bias (e.g., can we reduce the noise to +/- 
      10\% of expected)?  
    
    Can empirical correction models be developed on a site-level basis that 
      accounts for noise or bias?
  
  The approach is meant to characterize existing noise/bias but also outline 
    strategies for obtaining estimates of wq trends with more accuracy -- What 
    kinds of strategies give us high quality information? 
\end{comment}

Ecosystem metabolism is broadly defined as the difference between primary production and aerobic respiration and provides a basis for evaluating trophic state \citep{Kemp12,Needoba12}.  Primary producers, such as phytoplankton and vascular macrophytes, support and establish the means of energy transfer to upper trophic levels. Productive systems are characterized by more efficient transfer of organic matter between trophic levels, whereas less productive systems are sinks of organic matter that are supported by allochthonous sources of energy input.  The balance between production and respiration is an integrated measure of metabolism that accounts for varying rates in processes that create and consume organic matter.  Although metabolic rates vary naturally in different regions \citep{Caffrey04}, human activities and infrastructure development are contributing factors that increase rates of production \citep{Diaz08}.  Inputs of limiting nutrients beyond background concentraitons may stress the resilience an ecosystem such that higher rates of production are often associated with higher biological oxygen demand \citep{Yin04,Kemp09}.  Cultural eutrophication is frequently linked to declines in water quality through lower levels of dissolved oxygen and increased frequency of noxious algal blooms.  Reliables estimates of ecosystem metabolism are critical for measuring both background rates of production and potential impacts of human activities on ecosytem condition.     

Ecosystem metabolism can be estimated using several techniques, each of which is appropriate under a different conditions or assumptions.  Approaches can be generalized as one of two techniques.  Bottle-based techniques rely on rate measurements from discrete water quality samples, whereas open-water techniques infer metabolic rates using \textit{in situ} measurements from continuous monitoring data.  Bottle-based techniques are useful for direct partitioning of metabolic contributions into discrete habitats, such as planktonic production rates during specific time periods \citep{Kemp12}.  However, such measurements may be inappropriate for evaluating whole ecosystem metabolism if significant production occurs in other habitats, such as benthic or seagrass production.  As such, the open-water technique is an integrative measure of metabolism by inferring process rates from \textit{in situ}, continuous monitoring data.  Originally proposed for use in streams \citep{Odum56}, the method has been used with varying success in lakes \citep{Staehr10,Coloso11,Batt12}, and estuaries \citep{Caffrey04,Hopkinson05,Caffrey13}.  The ability of the open-water method to accurately estimate metabolism depends on whether the assumptions for its use are met, which are often only implicity verified in practiced. 

The open-water method uses the diel fluctuation of dissolved oxygen to infer rates of ecosytem metabolism, after correcting for losses or gains through air-water exchange \citep{Kemp12}.  Daily integrated measurements of metabolism are based on the balance between daytime estimates of gross production and nighttime estimates of respiration extrapolated to a 24 hour period.  The fundamental assumption of the open-water method is that measurements come from a water mass that has the same recent history \citep{Needoba12}.  Estimates of metabolism from a single location may be inaccurate if substantial variation in water column mixing occurs throughout the period of observation.  As such, the original technique designed for use in streams requires the comparison of data from an upstream and downstream station \citep{Odum56}.  Application of the method to systems without continuous flow, such as lakes or estuaries, have often assumed that a single sampling station provides sufficient data for estimating metabolism \cite{Staehr10}.  While single stations may be valid under specific conditions, numerous studies have shown that the open-water method may be inappropriate given the effects of physical mixing \citep{Ziegler98,Caffrey03,Coloso11,Batt12,Nidzieko14}.

The open-water method has recently been applied to coastal and oceanic ecosystems with mixed success.  An exhaustive analysis by \citet{Caffrey03} applied the method to estimate metabolim at 28 continuous monitoring stations at 14 US estuaries.  Data from two of the reserves were used to evaluate the assumption of homogeneity of water masses measured by each sensor.  Although significant differences were not observed for metabolism estimates between adjacent stations, the analysis was based on a comparison of means using conventional significance tests rather than a systematic comparison of time series.  Moreover, a portion of metabolism estimates from all stations were negative for production during the day and positive for respiration during the night.  These values were opposite in sign than expected since respiration consumes oxygen at night (i.e., negative effect on metabolism) and production increases oxygen during the day (i.e., positive effect on metabolism).  These `anamolous' values were attributed to violation in the assumption of water-column heterogeneity.  In particular, tidal variation could have caused sampling of different water masses by individual water quality sondes as water moved inland or seaward with changing tide. 

The effects of tidal advection on estimates of ecosystem metabolism have been a point of concern in numerous studies, although systematic estimates of its effects and methods for accounting for physical variation in \ac{DO} measurements have been minimal.  An exception is presented by \citet{Nidzieko14} through quantitative assessment of the effects of fortnightly tidal modulations on metabolism estimates.  Using a control volume approach to measure fluxes into and out of a shallow tidal creek, significant biases in metabolism estimates were observed.  Net heterotrophy was observed during spring tides, whereas metabolism was balanced during neap tides.  The timing of irradiance relative to the tidal cycle was a primary factor contributing to heterotrophy during summer months such that maximum tides occurred during the night increasing total area for respiration.  The results of the analysis, although specific to the study location, suggest that the effects of tidal advection on \ac{DO} measurements are of primary concern when selecting locations and length of time for sonde deployment in estuaries.  In many cases, the relative magnitude of these effects may be a significant source of bias without quantitative evaluation to determine the roles of biological and physical signals in \ac{DO} measurements. Analytical techniques to evaluate and correct for tidal advection could improve certainty in metabolism estimates and also increase the use of data from shorter deployment periods if sources of bias are quantified and removed.       

This article describes use of a novel method for quantifying and removing noise in estimates of ecosystem metabolim for estuaries.  Specifically, we characterize the effects of tidal advection on \ac{DO} observations to improve estimates of open-water metabolism with multi-year time series of high frequency ($<$ one hour) water quality data.  The focus of our analysis is the use of a weighted regression method previously developed for trend analysis of pollutant concentrations in streams and rivers.  The weighted regression approach is applied to create dynamic predictions of \ac{DO} as a function of time and tidal height change, which is then used to normalize, or detide, the \ac{DO} signal.  The analysis is presented in two steps.  First, we apply a simulation approach to create time series of \ac{DO} observations with known characteristics to evaluate ability of the weighted regression to predict the time series and remove the effects of tidal advection.  Second, four case studies of multi-year time series are used to further explore use of the weighted regression approach to remove potential noise in \ac{DO} signals from tidal advection.  Comparisons of observed and detided \ac{DO} values are compared, in addition to estimates of open-water metabolism before and after detiding of the \ac{DO} time series are obtained for each case study.  Overall, the analysis provides a means to improve certainty in conclusions from observed DO time series for evaluating the relative roles of biological and physical processes in estuarine systems.  Applications of the weighted regressoin approach are expected to have wide-ranging implications for management and ecosystem monitoring by explicitly addressing the assumption of water-column mixing in estimates of ecosystem metabolism.
