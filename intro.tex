Ecosystem metabolism describes the balance between production and respiration processes that create and consume organic matter \citep{Kemp12,Needoba12}.  Light exposure experiments of water samples collected at discrete locations and times have traditionally been used to measure metabolic activity.  Although highly controlled and precise, bottle-based methods are labor-intensive and not scalable to describe entire ecosystem rates.  Bottle-based methods may also only reliably estimate production and respiration associated with planktonic processes, whereas significant contributions of ecosystem production can arise from other habitats, such as benthic seagrass communities.  By contrast, open-water techniques have been increasingly used to estimate whole system metabolism given the availability of long-term, continuous time series of dissolved oxygen \citep{Odum56,Davanzo96}. Daily integrated measurements of metabolism represent the balance between daytime production and nighttime respiration attributed to all ecosystem components.  Open-water estimates also provide a basis for tracking ecosystem change over time and are more practical for capturing events or evaluating trends as compared to bottle-based approaches. Although metabolic rates vary naturally at different spatial and temporal scales \citep{Ziegler98,Caffrey04,Russell07}, anthropogenic nutrient sources are often contributing factors that increase rates of production (\citealt{Nixon95}, \citetalias{NRC00}).  Reliables estimates of whole ecosystem metabolism are critical for measuring both background rates of production and potential impacts of human activities on ecosytem condition.     

The ability to accurately estimate whole system metabolism using the open-water method depends on the degree to which assumptions of the theory are met \citep{Staehr10,Kemp12}.  The fundamental assumption is that the time series of \ac{DO} represents a Lagrangian specification of the flow field that describes the same water mass over time \citep{Needoba12}.  The Lagrangian specification assumes that the time series characterizes individual fluid parcels regardless of location, as in a parcel of water moving with the tide.  In reality, most \ac{DO} time series are collected at fixed locations such as a mooring or dock, which is characteristized by an Eulerian specification of the flow field.  Time series at fixed locations may characterize water masses with different metabolic histories if water particles are transported by physical advection. A Lagrangian flow field is often assumed, such that estimates of metabolism may be inaccurate if substantial variation in water column mixing occurs throughout the period of observation \citep{Kemp80,Russell07}.  Given this critical challenge, the open-water method has been used with varying success in estuaries influenced by tidal mixing \citep{Caffrey04,Russell07,Caffrey14}.  Appropriate placement of monitoring sondes, sampling frequency and duration, and reliability of data from single stations have been relevant issues in applying the open-water method to systems influenced by physical mixing \citep{Russell07,Staehr10}.  Application of the method to estuaries is a particular concern as physical mixing caused by tidal currents may confound the biological variation in \ac{DO} time series \citep{Kemp80,Caffrey03,Nidzieko14}.  Individual sampling stations near bay inlets or along major tidal axes may produce \ac{DO} time series that fail to meet the assumptions of the open-water method.   

Although numerous studies have shown that application of the open-water method to lakes or estuaries may be problematic \citep{Ziegler98,Caffrey03,Coloso11,Batt12,Nidzieko14}, very few quantitative approaches have been developed to address potential bias or noise in \ac{DO} signals from physical advection.  For example, an extensive analysis by \citet{Caffrey03} applied the open-water method to estimate metabolism at 28 continuous monitoring stations at 14 US estuaries.  A significant portion of the production and respiration estimates were negative (3 - 69\% depending on site), suggesting advection of water masses was a likely factor influencing the \ac{DO} time series.  These `anomalous' values are typically omitted from the analysis \citep{Caffrey03,Collins13}, which may upwardly bias estimates of metabolism \citep{Caffrey14}.  Further, \citet{Nidzieko14} evaluated the effects of tidal advection on metabolism estimates in a mesotidal estuary.  Estimates from a single location were strongly correlated with the spring-neap cycle such that net heterotrophy was more common during spring tides, whereas metabolism was generally balanced during neap tides.  A control-volume approach was used by impounding a section of the upper estuary to understand how physical processes contribute to biological variability.  Although useful as an \textit{in situ}, site-specific approach, more accessible statistical methods specific to time series are needed given the increasing availability of continuous monitoring data. For example, \citet{Batt12} explored the use of a Kalman filter \citep{Harvey89} to remove process and observation uncertainty from \ac{DO} time series in lakes.  Similar approaches have not been developed for estuaries, particularly those that address potential effects of tidal advection.

This article describes the development and application of a method for improving estimates of ecosytem metabolism computed from \ac{DO} time series.  Specifically, the apparent effects of tidal advection on \ac{DO} observations are removed to improve the fidelity of open-water metabolism estimates derived from continuous water quality data.  We used a weighted regression approach originally developed to resolve trends in pollutant concentrations in streams and rivers \citep{Hirsch10}.  The weighted regression approach creates dynamic predictions of \ac{DO} as a function of time and tidal height change, which are then used to filter, or detide, the \ac{DO} signal.  The model is based on the recognition that daily fluctuations in \ac{DO} are caused by metabolism associated with the solar cycle, whereas other fluctuations in estuaries are likely associated with water level changes that generally exhibit pregression relative to the solar cycle.  The weighted regression model was applied, rather than methods commonly used for detiding in physical oceanography, to allow for the complex and dynamic patterns of \ac{DO} changes relative to advection.  First, we used simulated \ac{DO} time series with known characteristics to evaluate ability of the weighted regression to remove the simulated effects of a tidally-advected \ac{DO} gradient.  Second, the simulation results informed the application of the method to four case studies chosen from the \aclu{NERRS} (\acs{NERRS}, \citealt{Wenner04}).  In all examples, tidal height is used as a proxy for lateral water movements that may influence \ac{DO} observations.  In the absence of quantitative data describing lateral \ac{DO} variation (e.g., contemporaneous stations along a tidal axis), we assume tidal height is an appropriate characterization of lateral variation.  Accordingly, `tidal variation' or `changes in tidal height' are used throughout in reference to assumed lateral \ac{DO} gradients that are carried past monitoring sensors by tidal currents.
