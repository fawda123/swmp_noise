\documentclass[letterpaper,12pt]{article}\usepackage[]{graphicx}\usepackage[]{color}
%% maxwidth is the original width if it is less than linewidth
%% otherwise use linewidth (to make sure the graphics do not exceed the margin)
\makeatletter
\def\maxwidth{ %
  \ifdim\Gin@nat@width>\linewidth
    \linewidth
  \else
    \Gin@nat@width
  \fi
}
\makeatother

\definecolor{fgcolor}{rgb}{0.345, 0.345, 0.345}
\newcommand{\hlnum}[1]{\textcolor[rgb]{0.686,0.059,0.569}{#1}}%
\newcommand{\hlstr}[1]{\textcolor[rgb]{0.192,0.494,0.8}{#1}}%
\newcommand{\hlcom}[1]{\textcolor[rgb]{0.678,0.584,0.686}{\textit{#1}}}%
\newcommand{\hlopt}[1]{\textcolor[rgb]{0,0,0}{#1}}%
\newcommand{\hlstd}[1]{\textcolor[rgb]{0.345,0.345,0.345}{#1}}%
\newcommand{\hlkwa}[1]{\textcolor[rgb]{0.161,0.373,0.58}{\textbf{#1}}}%
\newcommand{\hlkwb}[1]{\textcolor[rgb]{0.69,0.353,0.396}{#1}}%
\newcommand{\hlkwc}[1]{\textcolor[rgb]{0.333,0.667,0.333}{#1}}%
\newcommand{\hlkwd}[1]{\textcolor[rgb]{0.737,0.353,0.396}{\textbf{#1}}}%

\usepackage{framed}
\makeatletter
\newenvironment{kframe}{%
 \def\at@end@of@kframe{}%
 \ifinner\ifhmode%
  \def\at@end@of@kframe{\end{minipage}}%
  \begin{minipage}{\columnwidth}%
 \fi\fi%
 \def\FrameCommand##1{\hskip\@totalleftmargin \hskip-\fboxsep
 \colorbox{shadecolor}{##1}\hskip-\fboxsep
     % There is no \\@totalrightmargin, so:
     \hskip-\linewidth \hskip-\@totalleftmargin \hskip\columnwidth}%
 \MakeFramed {\advance\hsize-\width
   \@totalleftmargin\z@ \linewidth\hsize
   \@setminipage}}%
 {\par\unskip\endMakeFramed%
 \at@end@of@kframe}
\makeatother

\definecolor{shadecolor}{rgb}{.97, .97, .97}
\definecolor{messagecolor}{rgb}{0, 0, 0}
\definecolor{warningcolor}{rgb}{1, 0, 1}
\definecolor{errorcolor}{rgb}{1, 0, 0}
\newenvironment{knitrout}{}{} % an empty environment to be redefined in TeX

\usepackage{alltt}
\usepackage[top=1in,bottom=1in,left=1in,right=1in]{geometry}
\usepackage{setspace}
\usepackage[colorlinks=true,urlcolor=blue,citecolor=blue,linkcolor=blue]{hyperref}
\usepackage{indentfirst}
\usepackage{multirow}
\usepackage{booktabs}
\usepackage[final]{animate}
\usepackage{graphicx}
\usepackage{verbatim}
\usepackage{rotating}
\usepackage{tabularx}
\usepackage{array}
\usepackage{subfig} 
\usepackage[noae]{Sweave}
\usepackage{cleveref}
\usepackage[figureposition=bottom]{caption}
\usepackage{paralist}
\usepackage{acronym}
\usepackage{outlines}

%acronyms
% \acrodef{}{}

%knitr options


\setlength{\parskip}{5mm}
\setlength{\parindent}{0in}
\IfFileExists{upquote.sty}{\usepackage{upquote}}{}
\begin{document}
\raggedright

% \title{}
% \author{}
% \maketitle

{\it Response to review from Dr. Erik Smith, ``Improving estimates of ecosystem metabolism from dissolved oxygen time series'', by M. W. Beck, M. C. Murrell, and J. D. Hagy III}

Line by line response to reviewer comments are provided in italics.  Line numbers refer to the original manuscript. 

This manuscript presents a technique that attempts to remove the influence of advective oxygen transport on the estimation of net ecosystem metabolism rates computed from time series of dissolved oxygen (DO). There are a growing number of in situ data sondes (equipped with the new and robust optical DO sensors) being continuously deployed in coastal waters throughout the world. The use of these time-series to quantify and investigate the magnitude and variability in ecosystem metabolic rates continues to be an active area of research, having relevance to both basic and applied science questions. Of course a major challenge for the approach in coastal and estuarine environments is that, unlike lakes, tidal advection can greatly influence DO dynamics. As such, there is a great need for tools that can help improve estimates of net ecosystem metabolism from DO time series that include DO variability associated with tidal advection. The weighted-regression technique presented in this manuscript clearly represents a significant step forward in addressing this need this need and should make a significant contribution to the field.

While the approach presented does not entirely solve this challenge in all estuarine time series, I greatly appreciated that the manuscript clearly described how to determine from the method whether specific DO time-series are simply not amenable to net ecosystem metabolism computations (even with the approach presented here). That too is extremely valuable. 

While I had to read this manuscript a few times to really feel comfortable with it, I must say that I found it relatively easy to follow despite being very computationally ‘heavy.’ Of course, I still would not have the ability to do any of the computations needed to apply the technique on my own after reading this manuscript, but the manuscript certainly succeeded in conveying the need for the technique and what it can/can’t do if properly applied.

A couple of issues/questions I still have after digesting the manuscript:

1. From discrete measurements in the North Inlet estuary, metabolic rates, especially respiration, tend to be significantly higher on ebb tides, compared to flood tides (due to ebb waters being enriched in substrates, compared to flood waters). I am sure this holds for many estuaries, as well. I still cannot figure out if the technique presented allows for biological rates to vary over the course of tidal cycles or how this affects choice of averaging windows and resulting tidallycorrected DO signals. That is, if the time series includes a DO component produced by biologically-driven DO change that vary as a function of tide (on top of the changes that vary as a function of the PAR cycle) in addition to the physically driven tidal changes in DO, what is there an effect on the filtered time series and resulting NEM calculations?

{\it Our model is best suited to evaluate tidal effects occuring over a day or longer since the resulting metabolism estimates are daily values. It's likely that short-term variation in true metabolism that is related to the tide cannot be evaluated using the current techniques.  A broader discussion that should occur before applying the model is the extent to which the DO signal is affected by both physical advection and tidally-influenced metabolism.  Locations where tidal influences are large and the water is relatively deep (as in the Vierra Mouth station at Elkhorn Slough) are clear examples where physical movement is likely the dominant source of variation in the DO signal.  The method has real value in such instances.  Application to other stations where this is less clear should be carefully evaluated, particularly with respect to changing window widths on the results.  The following was added to the comments/recommendations section.

`DO time series may include components of metabolism related to true biological effects linked to the tide, in addition to diurnal cycling and effects of water movement.  For example, increased respiration may be observed during ebb tides as a result of increased enrichment of substrates (Sasaki et al. 2009).  In such cases, variation in the metabolic signal is correlated and directly related to the tide but is not related to physical movement of water masses. Accordingly, the model should not be used without critical evaluation of site-level characteristics that suggest physical movement of water masses is a primary source of noise in a DO signal.  Locations where tidal influences are large and the water is relatively deep (as in the Vierra Mouth station at Elkhorn Slough) are clear examples where physical movement is likely the dominant source of variation in the DO signal.  Application to stations where this is less clear should be carefully evaluated, particularly with respect to changing window widths on the results...' 

}

2. I am not sure I follow the reasoning behind assuming that reduced variation of metabolism estimates after the filtering procedure provides greater confidence of the estimates (line 428 and elsewhere). Substantial short-term variability in Pg and R are well documented from discrete measurements and perhaps this variability is just being “filtered out” by the procedure.

{\it This is a valid concern that we discussed at length during the development of this technique.  This is a potential issue when the solar cycle is correlated with tidal height changes such that the two are quantitatavely indistinguishable.  Figure 9 provides a visualization of these occurrences and when the method may not work as expected. Variation in the DO time series related to biology may be removed when the two are correlated.  When the two are not correlated, the model is unlikely to remove variation attributed to true metabolism.  In both cases, a reduction in variation would be expected and this may not be the most appropriate metric of performance. This is why we relied on four different performance measures. We feel that the comments and recommendations section adequately addresses these concerns.  The following was also added.

Line 346: `However, reduction in variation may also occur if the biological signal is reduced individually or in addition to physical variation.  The extent to which this reduction is related to the former should be minimal, provided that the two are statistically distinguishable.  Situations when the phasing of the tidal and solar cycles are correlated could be instances when the two are unable to be separated by the model.  Reductions in annual mean values (particularly at Elkhorn Slough) in addition to variance reduction suggests that true metabolism may have been removed causing a downward bias in the estimates.'
}

3. I had a bit of trouble following the “weight-of-evidence” approach for selection of halfwindow widths described starting on line 301. Not sure I fully understand / was able to follow the rationale for the performance metrics. Perhaps just a bit more detail on the thinking behind these metrics and how they specifically inform/influence the choices made?

{\it The following was added to the paragraph to more clearly describe selection of half-window widths using the performance metrics.  We also removed any reference to `weight-of-evidence' as this was an inaccurate description of the process.

`The selection of half-window widths for filtering the DO time series was based on an evaluation of results using four performance metrics.  The `optimal' window widths for each case study were those that provided the greatest measure of performance based on all four metrics.  First, the regression results were evaluated using correlations of DO and metabolism estimates with tidal height before and after application of the model, such that window widths that provided maximum reduction in correlation relative to the observed data were desirable [...]. The second and third performance metrics evaluated changes in the annual mean metabolism estimates and standard deviation before and after filtering [...] Optimum window widths in this context were considered those that maintained the mean values while reducing standard deviation relative to the observed data. Finally, results were evaluated based on the occurrence of `anomalous' daily production or respiration estimates [...] Optimum window widths for this metric were those that provided the maximum reduction in anomalous values.'
}

Some minor specific comments/suggestions:

The title is a bit vague and does not really speak to what is novel about the work. Might think about including specific mention of correcting for tidal advection.

{\it The title was modifed: Improving estimates of ecosystem metabolism by reducing effects of tidal advection on dissolved oxygen time series}

Much of the introductory paragraph does not seem particularly germane to the focus of the paper. I might suggest what is needed to start the Introduction is an argument for why estimates of Pg, Rt and NEM by the open-water technique are advantageous despite caveats and uncertainties of the method (e.g., high resolution sampling captures events missed by traditional grab sampling methods; bottle measurements are labor intensive and have their own caveats; NEM captures entire ecosystem processes rather than components; etc.)

{\it The introductory paragraph was rewritten to emphasize the above points.

`Ecosystem metabolism describes the balance between production and respiration processes that create and consume organic matter (Kemp and Testa 2012, Needoba et al. 2012).  Light exposure experiments of water samples collected at discrete locations and times have traditionally been used to measure metabolic activity.  Although highly controlled and precise, bottle-based methods are labor-intensive and not scalable to describe entire ecosystem rates.  Bottle-based methods may also only reliably estimate production and respiration associated with planktonic processes, whereas significant contributions of ecosystem production can arise from other habitats, such as the benthos or seagrass patches.  By contrast, open-water techniques have been increasingly used to estimate whole system metabolism given the availability of long-term, continuous time series of dissolved oxygen (Odum 1956, D'Avanzo et al. 1996). Daily integrated measurements of metabolism represent the balance between daytime production and nighttime respiration attributed to all ecosystem components.  Open-water estimates also provide a basis for tracking ecosystem change over time and are more practical for capturing events or evaluating trends as compared to bottle-based approaches. Although metabolic rates vary naturally at different spatial and temporal scales (Ziegler and Benner 1998, Caffrey 2004, Russell and Montagna 2007), anthropogenic nutrient sources are often contributing factors that increase rates of production (Nixon 1995, NRC 2000).  Reliables estimates of whole ecosystem metabolism are critical for measuring both background rates of production and potential impacts of human activities on ecosytem condition.'
}

Line 49: I would also cite Staehr et al. 2010 paper along with the Kemp and Testa 2012 reference, because it is from the same journal this manuscript will be submitted to, it is a really nice review of the concepts and method details (even if specific to lakes), and it is much more accessible than the book that has the Kemp and Testa reference.

{\it Citation was added.}

Lines 61-72: I think that the method has been applied rather successfully to lakes for many years, mainly because advection is not a significant problem in lakes. Where the method has problems in lakes is when vertical stratification/stability is an issue. In contrast the greatest difficulty in applying the method to coastal waters is the problem of advection, such that the method has relatively rarely been applied in estuaries (with Caffrey being the exception).

{\it Lines 61-64 were revised to reflect these points.

`Given this critical challenge, the open-water method has been used with varying success in estuaries influenced by tidal mixing (Caffrey 2004, Russell and Montagna 2007, Caffrey et al. 2014).  In contrast, the method has been more successfully applied to water quality time series in lakes, although stratification may limit estimates to specific vertical layers (Staehr et al. 2010, Coloso et al. 2011, Batt and Carpenter 2012).'
}

Lines 142-144: These two sentences need verbs.

{\it Added verbs to these sentences:

`Windows for time and hour are used to weight observations based on distance (time) from the center of the window.  The window for tidal height is used to weight observations based on the difference from the center as a proportion of the total tidal height range.'
}

Line 298: Don’t you mean that respiration at night was assumed to be equal to respiration during the day, rather than respiration was assumed constant during the “night”

{\it Sentence was revised:

`Respiration rates were assumed constant during day and night such that total daily rates were calculated as hourly respiration multiplied by 24. The ``metabolic day'' was considered the approximate 24 hour period between sunsets on two adjacent calendar days. Respiration was subtracted from daily net production estimates to yield gross production.'
}

Line 313: “gross production” rather than “production” just to be explicit.

{\it Text was changed.}

\end{document}
