\documentclass[letterpaper,12pt]{article}\usepackage[]{graphicx}\usepackage[]{color}
%% maxwidth is the original width if it is less than linewidth
%% otherwise use linewidth (to make sure the graphics do not exceed the margin)
\makeatletter
\def\maxwidth{ %
  \ifdim\Gin@nat@width>\linewidth
    \linewidth
  \else
    \Gin@nat@width
  \fi
}
\makeatother

\definecolor{fgcolor}{rgb}{0.345, 0.345, 0.345}
\newcommand{\hlnum}[1]{\textcolor[rgb]{0.686,0.059,0.569}{#1}}%
\newcommand{\hlstr}[1]{\textcolor[rgb]{0.192,0.494,0.8}{#1}}%
\newcommand{\hlcom}[1]{\textcolor[rgb]{0.678,0.584,0.686}{\textit{#1}}}%
\newcommand{\hlopt}[1]{\textcolor[rgb]{0,0,0}{#1}}%
\newcommand{\hlstd}[1]{\textcolor[rgb]{0.345,0.345,0.345}{#1}}%
\newcommand{\hlkwa}[1]{\textcolor[rgb]{0.161,0.373,0.58}{\textbf{#1}}}%
\newcommand{\hlkwb}[1]{\textcolor[rgb]{0.69,0.353,0.396}{#1}}%
\newcommand{\hlkwc}[1]{\textcolor[rgb]{0.333,0.667,0.333}{#1}}%
\newcommand{\hlkwd}[1]{\textcolor[rgb]{0.737,0.353,0.396}{\textbf{#1}}}%

\usepackage{framed}
\makeatletter
\newenvironment{kframe}{%
 \def\at@end@of@kframe{}%
 \ifinner\ifhmode%
  \def\at@end@of@kframe{\end{minipage}}%
  \begin{minipage}{\columnwidth}%
 \fi\fi%
 \def\FrameCommand##1{\hskip\@totalleftmargin \hskip-\fboxsep
 \colorbox{shadecolor}{##1}\hskip-\fboxsep
     % There is no \\@totalrightmargin, so:
     \hskip-\linewidth \hskip-\@totalleftmargin \hskip\columnwidth}%
 \MakeFramed {\advance\hsize-\width
   \@totalleftmargin\z@ \linewidth\hsize
   \@setminipage}}%
 {\par\unskip\endMakeFramed%
 \at@end@of@kframe}
\makeatother

\definecolor{shadecolor}{rgb}{.97, .97, .97}
\definecolor{messagecolor}{rgb}{0, 0, 0}
\definecolor{warningcolor}{rgb}{1, 0, 1}
\definecolor{errorcolor}{rgb}{1, 0, 0}
\newenvironment{knitrout}{}{} % an empty environment to be redefined in TeX

\usepackage{alltt}
\usepackage[top=1in,bottom=1in,left=1in,right=1in]{geometry}
\usepackage{setspace}
\usepackage[colorlinks=true,urlcolor=blue,citecolor=blue,linkcolor=blue]{hyperref}
\usepackage{indentfirst}
\usepackage{multirow}
\usepackage{booktabs}
\usepackage[final]{animate}
\usepackage{graphicx}
\usepackage{verbatim}
\usepackage{rotating}
\usepackage{tabularx}
\usepackage{array}
\usepackage{subfig} 
\usepackage[noae]{Sweave}
\usepackage{cleveref}
\usepackage[figureposition=bottom]{caption}
\usepackage{paralist}
\usepackage{acronym}
\usepackage{outlines}
\usepackage{amsmath}

%acronyms
% \acrodef{}{}

%knitr options


\setlength{\parskip}{5mm}
\setlength{\parindent}{0in}

\newcommand{\Bigtxt}[1]{\textbf{\textit{#1}}}
\IfFileExists{upquote.sty}{\usepackage{upquote}}{}
\begin{document}
\raggedright

% \title{}
% \author{}
% \maketitle

{\it Response to reviewer comments ``Improving estimates of ecosystem metabolism from dissolved oxygen time series'', by M. W. Beck, M. C. Murrell, and J. D. Hagy III}

{\it The authors wish to thank the reviewers and associate editor for providing thoughtful comments on our manuscript.  We have provided our response to each of these comments below, indicated in italics.  Line numbers refer to the original manuscript. The reviewer comments have been shortened for brevity.}

\Bigtxt{Reviewer 1:}

I hope the authors can address this difficult question: does the model work better in diurnal tidal systems because the errors between the day and night conditions on the end-members cancel each other out? This is as opposed to the mixed tides where the errors can be heavily skewed towards respiration or production during the high high tide and thus lead to the large inaccuracies noted for Elkhorn Slough. This is an important point, because if the weighted regression approach is generating offsetting errors then it is not really working.

I think that the caveats described above have been acknowledged by the authors, such as starting on line 527, but I would recommend framing the issues associated with the “end member” problem in a very straightforward way. 

{\it We suggest that the model can work equally well regardless of tidal characteristics, provided appropriate window widths are used and collinearity between tidal and solar cycles is minimal.  These points were previously mentioned (e.g., lines 431, 521) but additional explanation is needed based on the reviewer's comments.  We have added text to this section to emphasize how and why the model can effectively characterize and remove the tidal signal, emphasizing that the model can work equally well regardless of tidal regime. 

`Understanding physical advection in estuaries and how movement varies by tidal regime can help conceptualize how and why weighted regression reduces tidal effects on DO time series.  Noise in the biological signal caused by longitudinal movement of water past a sensor can be differently characterized depending on tidal regime and habitat characteristics of the estuary.  Sites with mixed tides can have prolonged periods up to several weeks where tidal maxima occur during periods of peak respiration or production.  For example, Nidzieko et al. (2014) and the current analysis have shown that high tide at Elkhorn Slough during summer months more often occurs during nighttime leading to overestimates of respiration.  This contrasts with semidiurnal or diurnal tides that pregress more rapidly with the solar cycle such that the effects of physical advection are more apparent on shorter time scales.  The metabolic history of water masses occurring at the `end members' of the tidal excursion also influences the relative noise in the DO signal.  Time series at sites within a tidal excursion that includes a mosaic of habitats (e.g., open ocean, seagrass beds, tidal marshes) will have a stronger signal of physical advection.  Choosing appropriate window widths for a given site depends entirely on the tidal regime and the types of habitat within an estuary.  Extreme examples, such as macrotidal sites with mixed tides and different habitat types, will require larger window widths to isolate the true metabolic signal as the moving regression for each observation must include sufficient observations to characterize the average.  Biological signals from diurnal or semidiurnal tides can be statistically isolated with relatively smaller window widths as the true metabolic signal can be quantified in fewer tidal cycles.  Weighted regression has the ability to describe the true metabolic signal regardless of tidal characteristics and biological processes that vary at the end members, although chosen window widths and collinearity with the solar cycle affect its application.'
}

Specific questions or corrections:

1) This may be a naive question -  but is tidal height an accurate proxy for advection? It is intuitive that the higher the tidal amplitude the more advection that will occur over the period. However, it is also true that the fastest currents associated with tides occur in the mid-tide range, for example on a large ebb tide the highest currents (and presumably a significant part of the advection term) occur when the tide is actually near the mid-range in tide height. Equation 13 seems to contradict this fact, where the highest advection (DOadv) is associated with the highest tide measurement (H). This is in fact slack tide when very little advection is occurring.

{\it We recognize that tide height is not the same measure as advection.  Weighted regression only requires a variable that is statistically correlated to advection.  The model is effective because it isolates the changes in DO to tidal height within the moving window, such that actual physical differences between tidal height and advection are inconsequential.  We have added additional text to emphasize this distinction, beginning on line 123.  We have also revised lines 204-213 in the methods to clarify differences between the simulated data and actual time series.

Line 123: `An important distinction between tidal height and advection is that each variable could provide different information about tidal regime.  For example, tidal heights at the minimum and maximum of the range for standing wave tides may be associated with periods of low advection when water masses are not moving rapidly past the sensor.  The relationship between tidal height and advection across multiple environments may be complex.  The weighted regression model is suited for application to different locations because it needs only a variable that indicates a particular point in the tidal cycle.  The specific influence of advection on the DO time series is isolated using the moving window approach such that the relationship between tidal height and advection does not need to be known.
'

Line 204: `A tidal series was simulated based on the principal lunar semidiurnal (M2) tide with a period of 12.42 hours (Foreman and Henry 1989).  The amplitude was set to 1 meter and centered  at 4 meters.  The tidal time series simulated DO changes with advection, $DO_{adv}$ (eq. (7) and Fig. 1). Conceptually, this vector represents the rate of change in DO as a function of horizontal water movement from tidal advection such that:
\begin{equation} \tag{10}
\frac{\delta DO_{adv}}{\delta t} = \frac{\delta DO}{\delta x} \cdot \frac{\delta x}{\delta t}
\end{equation}
where the change in DO from advection over time $t$ is the product of the change in DO over the change in tidal excursion $x$ and the change in tidal excursion over time.  The simulated tidal time series was used to create $DO_{adv}$ as a simple sine wave as in eq. (8).  This time series was centered at zero to simulate increasing or decreasing DO with tidal height change.  The final time series for observed DO was the sum of biological DO and advection DO (eq. (4) and Fig. 1).  For actual time series, changes in DO from advection can be described generally as a function of tidal height:
\begin{equation} \tag{11}
DO_{adv} = f\left(H\right) 
\end{equation}
The functional relationship between tidal height and changes in DO from physical advection may vary depending on tidal regime, e.g., standing wave versus progress tides.  For weighted regression, all that is required is a variable that is statistically correlated with tide to quantify the influence on the DO time series.  Our simulated data assumes that tidal height is directly proportional to changes in DO and, therefore, statistically correlated.  This assumption may be invalid for actual time series but it is appropriate for evaluating the empirical qualities of the weighted regression model, as described in the previous section.'  

}

2) The symbol for uncertainty in equation (6) and (7) and in the text on page 11 is not the same as the symbol used in figure 1, and both are different than at the bottom of page 13.

{\it We have verified all equations, text, figures, and tables show the same same symbol, $\varepsilon$.}

3) Line 247: should this be decreased process error?

{\it No, we found that increasing process uncertainty actually improved the ability of the model to isolate the biological signal.  Process error is serially correlated and we suspect that this correlation structure is characterized well by the model since time is used as an explanatory variable.  This is a minor point because this effect was most pronounced when there was no advective component and observation error was high in the simulated DO time series.  In other words, there is no reason to apply weighted regression to an actual time series where advection effects are minimal.}

4) Wind speed can have a significant influence on the calculation of DO flux, and therefore should be acknowledged as another source of error not accounted for by using a constant ka value. This influence is in addition to the advection term and can be significant.

{\it Yes, our implementation of the open-water method for estimating metabolism accounts for changes in DO from wind.  This was not apparent in the previous text and we have added additional information.

Line 300: `This coefficient accounts for the influence of meteorological effects, such as wind, on air-sea gas exchange at the surface.'

Additionally, we did not explicitly consider any effect of wind in the simulated time series because it is not relevant for the analysis where the primary objective was to evaluate advective effects.  Wind effects could be considered a component of process uncertainty, which was previously included.  The following was added for clarification.

Line 186: `For example, wind events can affect air-sea gas exchange (Ziegler and Benner 1998; Caffrey et al. 2014) such that high wind may contribute to increased process uncertainty.  Although this was not an explicit focus of the simulations, wind effects could be considered an implicit component of process uncertainty in addition to the effects of other unmeasured or latent variables that influence DO in a time-dependent manner.'
}

5) The different scales on Fig 6 and 7 for DO are confusing. 

{\it The figures now have the same scale for DO.}

6) It would be beneficial to readers if the code for the analysis was distributed with the published paper.

{\it we created a simple R package on GitHub for implementing weighted regression and estimating ecosystem metabolism. A link was added as supplementary information: \href{https://github.com/fawda123/WtRegDO}{https://github.com/fawda123/WtRegDO}}  

\Bigtxt{Reviewer 2:}

Do the authors plan to make their R code available? They mention several times how useful it could be to others (a point on which I certainly agree), but it won’t be useful unless it’s available!

{\it We have added a link in the supplementary information for an R package to implement weighted regression and ecosystem metabolism: \href{https://github.com/fawda123/WtRegDO}{https://github.com/fawda123/WtRegDO}}

General comments:

(1) The new method offers significant advantages for the types of systems in the NERRS network, which are the focus of their study. The authors touch on this distinction briefly in lines 98-100, but I think they need to expand on the point and make it more prominent.

{\it NERRS sites are often characterized by heterogenous habitats in a complex mosaic, which simultaneously creates interesting research opportunities but also complicates evaluation of temporal variation of ecosystem characteristics.  As the reviewer notes, the weighted regression approach is ideal for such locations because it explicitly targets the tidal component using a dynamic fitting process.  This is in contrast which traditional deconvulution methods that are likely to produce suboptimal results in complex environments such as NERRS.  We have added text to emphasis the advantage of using weighted regression in estuarine environments.

Line 100: `The method targets the tidal component as an explicit variable using dynamic model parameters that change with location in the time series.  This allows an isolation of the biological component of the DO time series that is independent of advection.  As a result, the weighted regression approach can preserve the true biological signal rather than risking removal of both the biological and physical components as with traditional deconvulution methods.  This offers an advantage for estuaries where the magnitude of tidal effects on water quality observations can be larger than in offshore or coastal environments.'
}

(2) I was generally confused, both in the authors’ application of their filtering method, and in the way they calculated their metabolic estimates, about the role of the third dimension (depth). Did the authors apply the same weight vectors and half-window widths to the data from each depth in a given system? Or, perhaps the authors chose the four systems they did for their case studies because the depth at those particular stations was shallow enough such that only one series of DO data was needed? If the latter case is true (i.e., data from only one depth used at each station), how would the authors propose applying their method in a deeper water column? Could it be? 

{\it Yes, we assumed DO was vertically homogenous.  This is a reasonable assumption for the case studies since all NERRS sites are shallow, productive estuaries.  The method is then of course limited to only shallow water systems, but we argue for broad appeal as these systems are widespread across the globe, in addition to the 28 reserve systems within NERRS.  Additionally, extending the method to stratified systems could be possible although additional data would be needed and the implementation would be much more complex (i.e., accounting for pycnocline exchange rates, etc.).  The following was added to clarity thse points:

Line 259: `NERRS is a network of 28 shallow, productive estuary reserves...'

Line 285: `NERRS sites are typically shallow and vertically-mixed such that one water quality monitor is adequate for the entire water column, including the benthos.'

Line 445: `The regression method and open-water technique may also have broad appeal for application in estuaries that are shallow and vertically-mixed, such as those within NERRS. Extension to stratified systems may be possible with additional data and model validation.'   
}

Further:

(a) Unless I missed it, what was the method of integration by which the authors got from their volumetric estimates for each time step in equation (14) (specified in g O2 m-3 hr-1) to the depth- integrated (areal) estimates (in g O2 m-2 t-1) that appear in all of their figures and tables?

{\it Text was added for clarification on line 307: `Finally, volumetric rates were converted to depth-integrated (areal) estimates (g O$_2$ m$^{-2}$ d$^{-1}$) by multiplying by the mean water-column depth at the site.  Site depth was taken from the water quality sensor and a half-meter was also added to account for approximate placement slightly off of the bottom.' 
}

(b) I am concerned (or perhaps just confused) about the method the authors chose for deriving daily estimates of P and R from the hourly fluxes calculated using equation (14). The description of the method is not clear. Did the authors (1) calculate the individual hourly fluxes, then group them by “day” or “night,” then average them, and then multiply each by the duration of the day or night, or (2) calculate the individual hourly fluxes, then divide them by “day” or “night,” then add them together within their respective groups? Both methods have been used in the DO time- series metabolism literature; the second may be better since it doesn’t assume the rate of primary production is constant over the course of a given day.

{\it We used an approach identical to Caffrey et al. 2014 (cited on lines 294, 298) to average DO flux in each day/night period, which is then multiplied by different time periods to estimate production and respiration.  Fortunately, the two approaches as described would produce an identical result, which we confirmed using a simple spreadsheet comparison.  For the purpose of the comment, we have revised the text to clarify the method:

Line 300: `The diffusion-corrected DO flux estimates as hourly rates of DO change were first averaged during day and night periods for each 24 hour `metabolic day' in the time series. The `metabolic day' was considered the period between sunsets on two adjacent calendar days.  Respiration was hourly DO flux during night hours and net production was hourly DO flux during day hours.  Total respiration (Rt) rates were assumed constant during day and night such that daily rates were calculated as the average DO flux during night hours multiplied by 24. Daily gross production (Pg) was the average DO flux during day hours minus the average daily respiration, multiplied by total sunlight time.  Net ecosystem metabolism was gross production (positive) plus total respiration (negative).'
}

(3) Notation and terminology:

(a) The use of “Pg” to denote gross primary production is very confusing, especially since the units of Pg are in g (and Pg also being the SI unit for petagrams). Perhaps something else would be more appropriate? In addition, what is the “t” in “Rt” as the notation for respiration? Total? If so, the authors should specify when they introduce their variable (unless they did, and I missed it?).

{\it These are standard terms that describe gross production and total respiration (see Caffrey et al. 2014, references therein).  The above revision in response to the previous comment defines the acronyms accordingly.}

(b) The word “aggregation” is used throughout the text. I gather from my reading of the ms that the authors mean binning or averaging, over some time or spatial scale. “Aggregation” is not a particularly precise term, because it could mean many different things. Perhaps the authors could consider replacing this word in the ms with a word or phrase that describes what they mean, e.g. “time averaging.”

{\it We have noted all instances in the text where `aggregation' is used and revised for clarity, mostly replacing with `time averaging' or greater detail as needed.}

(4) Duplication of presentation. If space is an issue, it seems as though Figs. 3 and 4 present the same data as Tables 1 and 2. I know heatmaps are all the rage right now, but I honestly felt the specific correlation coefficients in the tables told me a lot more than the heatmaps.

{\it We are willing to remove these figures at the editor's request but we would prefer they be retained because they do not show identical information.  The figures provide an overview of the correlation and error associated with each unique combination of simulation conditions (Fig. 3) and window widths (Fig. 4).  The tables provide a distribution of the correlation and error for each unique parameter where a criteria applied.  For example, the first row in table 1 shows the distribution of the correlation/error for all simulation results when the diel DO component was set at zero.  This information is a summary whereas all values are shown in Fig. 3 as 27 tiles where the diel DO component was zero.

Text was added to figure/table captions to emphasize these differences.  For example, figure 3 caption now includes `See Table 1 for a summary of combined results for each unique parameter', whereas Table 1 now includes `See Fig. 3 for results of all parameter combinations.'
}

Specific comments:

Abstract generally: If the authors could work in a sentence on the significance of their method vis-à-vis traditional time-series filtering techniques (see my comments above), I think they could increase the impact of their work. Additionally, I would have liked to see at least a few numbers or figures in the abstract (e.g., by what \% on average did the method reduce variance in the NEM estimate relative to those calculated from the unfiltered data, or how much did it reduce RMSE?).

{\it The following was added to the abstract: `The method targets the tidal component using a flexible association while preserving the true biological signal, offering an advantage over traditional deconvulution methods.'

We have decided not to include results in the mansucript.  We feel that presenting simple summary numbers without adequate explanation would be more misleading than it is informative since the model did not work in all cases.  The abstract requirements for L\&O Methods is word-limited and we feel that we cannot adequately present this information within the allotted space.
}

p. 3, line 12: “integrated” by what dimension and over what interval? Depth? Time?

{\it This term was removed as the details are now available in the text.}

p. 3, line 22: “useful” is nonspecific and does the authors’ work a disservice; can they think of
something more specific and impactful?

{\it See our response to the first specific comment.  We have also revised some of the text for clarity: `Variability from advection was reduced in all cases, in addition to a reduction of the correlation with the tide. The model was especially effective when the magnitude of tidal influence was high and correlations between tidal change and the solar cycle were low, whereas collinearity in the latter instance limited model performance.'}

p. 3, lines 23-24: See my comment above regarding the method’s performance when the solar cycle and period of tidal height change were highly collinear. The method yielded estimates at the Elkhorn and Padilla sites which were significantly different from those obtained with the unfiltered (traditional) data. By the authors’ own admission, neither the filtered nor unfiltered data probably accurately represent what is truly happening at those sites. A sentence specifying those situations in which the method did not perform particularly well would be more honest, rather than simply saying when it generally did work well.

{\it The sentence was modified as follows: `The model was especially effective when the magnitude of tidal influence was high and correlations between tidal change and the solar cycle were low, whereas collinearity in the latter instance limited model performance.'
}

p. 3, line 26: “timescales”

{\it Changed.}

p. 4, lines 32-33: Reference to Gaarder and Gran (1927) missing here.

{\it Citation added.}

p. 4, lines 42-43: This simply isn’t true. There are dozens of reasons bottle incubations are more effective in certain situations, depending on the research objective. Open-water measurements may be more effective for evaluating certain kinds of events or trends, but if the authors are going to make this claim, they should specify what they mean.

{\it The sentence was revised: `Open-water estimates also provide a more accessible means of tracking ecosystem change over time as compared to discrete sampling events with bottle-based approaches.'}

p. 4, line 46: “Reliable estimates”

{\it Changed.}

p. 5, line 67: “advection, leading to…”

{\it Comma added.}

p. 6, line 81: “example illustrates”

{\it Changed.}

p. 7, lines 98-100: See my comments above about the method’s significance vis-à-vis traditional detrending methods. An expanded discussion of the difference here is warranted.

{\it See the response to the first general comment, i.e., the additions to the final paragraph of the introduction.}

p. 7, lines 101-102: “evaluate the ability”

{\it Changed.}

pp. 7-8: Appropriate place for clarifying role of depth, or why it is being ignored (my comment above).

{\it Yes, see the response to the second general comment.  Also, the response to the next comment addresses the role of tidal height/depth vs. advection as a model variable.}

p. 8, lines 121-122: I buy the idea, but is there some justification in the literature for using tidal height as a proxy for advection?

{\it See our response to reviewer 1 about the use of tidal height as a proxy for advection, i.e., the expansion of the paragraph on page 8... 'An important distinction between tidal height and advection...'}

p. 9, lines 139-140: I was generally confused about why different “time” vectors were used to represent days and hours (or is it “hour of day”) separately. Maybe just some clarification of terminology would help.

{\it The weighted regression model evaluates weights associated with distance from the center of the window in days as well as hours within the day.  For example, if we care about modelling oxygen at 3 pm, we would weight all other observations close to 3pm in other days with high importance, with further diminishing importance as window distance increases in days from the center.  The weights widget in the supplementary information can be used to view the influence of both, in addition to the tidal weights.  The citation at the end of the paragraph for the multimedia section should help interpretation.  We also revised the line... `Windows for time (continuous throughout the day) and hour (within each day) are used...'
}

p. 12, line 196: What is the justification for using this particular range of O2 concentrations to generate the simulated data set?

{\it The values were chosen based on a general approximation of the observed data from the case studies, text was added... `as a rough approximation from the case studies below'}

p. 12, line 204: “lunar”

{\it Changed.}

p. 13 ff: Equations appear to employ the wrong “del” for the partial derivative, i.e. versus

{\it This was changed to summation.}

p. 14, line 237: The authors are characterizing a corr. coefficient of 0.63 as “similar”? Or did I
miss something? On what specific basis in Tables 1 and 2 is the claim of similarity made?

{\it `Similar' may have been too strong of a description for these results.  Instead, we suggest that the filtered time series from simulation results produced a `reasonable representation' of the true biological signal for most scenarioes.  Poor results were only obtained for scenarios when advection was minimal and process uncertainty was high, which was previously noted on lines 450-451.}

p. 17, lines 299-300: Related to my comment about depth, above. To get from a piston velocity
to a “volumetric reaeration coefficient” (isn’t this just a depth-corrected piston velocity - why the need for even more jargon?), one must invoke some mixed layer depth. What was the depth (i.e., “H” in Thébault et al’s equation A11) used in each case? The Thébault paper seems to contain a not-so-unique method of calculating gas transfer velocity (and air-sea flux), gussied up with new terminology for some of the parameters.

{\it We have not changed the terminology to reduce confusion between our methods and those we cite.  This work builds on the metabolic approach described in Caffrey et al. 2014.  Ongoing work here at EPA and elsewhere will further develop these methods.  We anticipate that interested readers will consult our references for more details on the exact techniques.  Further, we have clarified the role of depth in the response to the previous comment about areal rates.}

p. 17, lines 300-307: The precise method of obtaining the daily ecosystem metabolism estimates was not clear. (See my general comment 2, above.)

{\it See our response to the second general comment.}

p. 18, lines 319-321: I think this assumption is valid, but can the authors provide some justification or precedent? One citation would be sufficient.

{\it We have unpublished data that generally support this claim, unpublished citation added.  To our knowledge, no studies have examined these relationships at the level of detail described in our manuscript so we cannot provide a peer-reviewed citation.}

p. 19, lines 350-351: I disagree. These percentages from the filtered data do represent a decrease
in the number of anomalous values by a factor of ~ 3, but they are not “near zero.” In one case,
7.12\% of values were still anomalous.

{\it Sentence was revised: `Anomalous values were substantially reduced for all case studies...'}

p. 23, line 427: Particularly confusing use of the word “aggregated.” Aggregated or averaged by
what, and over what time or spatial interval?

{\it Sentence was revised: `Results for Sapelo Island suggested that time averaged estimates, either as monthly or annual means, were comparatively unchanged by filtering...'}

Discussion generally: The discussion read less like a discussion than a simple summary of the rest of the paper. Perhaps a separate discussion isn’t really necessary?

{\it The author guidelines for L\&O methods indicate that the discussion should describe the degree to which the method meets defined needs, potential for new insight, comparison with alternative approaches, and new questions that may have been raised (\href{http://onlinelibrary.wiley.com/journal/10.1002/(ISSN)1541-5856/homepage/ForAuthors.html#6}{http://onlinelibrary.wiley.com/journal/10.1002/(ISSN)1541-5856/homepage/ForAuthors.html\#6}).  We argue that the current discussion adequately covers these points so we have not removed it from the manuscript. 
}

p. 23, lines 440-441: “remove some variation”; the method doesn’t remove all variation

{\it Changed, though we have used `a significant amount' instead of `some'.}

p. 26, lines 500-503: This seemed like it (or some similar text) belonged in the methods section, to describe why the sites were selected. In addition, the grammar of this sentence was confusing.

{\it Sentence was revised and moved to line 269 in the assessment section: `Many daily metabolism estimates were also not interpretable (i.e., `anomalous') from these sites using standard open water methods.'
}

p. 27, lines 527-535: This paragraph belongs in the discussion section. Perhaps after the paragraph ending on line 479.

{\it Moved.} 

p. 28: Will code be available publicly somewhere?

{\it Yes, see the comment above and in response to reviewer 1.}

Comments on figures and tables:

Figure 1 ff: One of my few questions about the simulation datasets centers on the way “biology” was parameterized. I appreciate that modulation in the DO signal due to biological processes was modeled as a purely sinusoidal function for purposes of evaluating the method, but even the purest biological signal will not be sinusoidal. Major differences are (1) that the peak of the biological curve will usually lag each day’s peak insolation by some number of hours (evident in the authors’ own data in Fig. 6) and that (2) there is a point of photoinhibition, where the perfect shape of the curve is “dented” by a decrease in photosynthetic activity before it begins its actual decline. I’ve not put either of these very eloquently, but I assume the authors know what I’m speaking of. Some brief acknowledgement in the methods section about the representation of the biological signal as a perfect curve like this would be instructive.

{\it Text was added in the methods to clarity this point (line 196): `Although the representation of $DO_{die}$ as a simple sine wave does not account for lags with the solar cycle or periods of photoinhibition that may be observed in actual time series, we consider the approach sufficient for the simulations.'}

Figure 6: What was the threshold PAR intensity used for defining the PAR shading? What are the units of PAR? Millmoles of what? Photons? Microeinsteins (µE)? Or W m-2?

{\it The shading indicates day or night based on astronomical sunrise/set times for the location and date.  This was clarified in the captions.  Units for PAR were also added ($\mu$E m$^{-2}$ s$^{-1}$).}
 
Figure 8 ff: One effect of filtering with the authors’ method is that it appears to dramatically attenuate the interday variation in metabolic parameters evident in unfiltered data. Remember that the “discovery” of this temporal heterogeneity has been one of most widely celebrated features to emerge from analysis of high-frequency time series (of, e.g., D.O.). What is the implication of the authors’ method for the existence (or not) of this heterogeneity? Does this mean there really isn’t as much temporal heterogeneity in ecosystem respiration and in photosynthesis as this method has suggested? Some discussion would be instructive (perhaps
replacing some of the repetitive text in the current discussion); the authors won’t have to look far into the literature to find many studies that have remarked on the apparent heterogeneity in unfiltered data.

{\it This figure shows the effect of filtering on metabolism for a two-week period at Sapelo Island, with substantial muting of variation in the filtered results.  Based on our understanding of water quality and habitat at Sapelo Island, we are unaware of any factors that could contribute to interday variation of the magnitude seen in the observed data.  This interday variation is as large as the seasonal variation in metabolism (see the multimedia widget for Sapelo Island  \href{https://beckmw.shinyapps.io/detiding_cases/}{https://beckmw.shinyapps.io/detiding\_cases/}) such that daily variation in metabolic rates from biological processes alone is unlikely.  This analysis provides evidence that calls into question the popular belief that interday variation in metabolism can be quite large.  This hypothesis should be further investigated and we are hopeful that the method in our manuscript could provide a means to better understand patterns in metabolism across multiple time periods.  We have added text in the discussion to promote these ideas (line 479).    

`These processes could contribute to interday variation in metabolic rates, although our results suggested that this variation may be reduced after removing tidal effects (Fig. 8).  In many cases, the interday variation observed in metabolism using the unfiltered data was as large or larger than the seasonal variation, indicating substantial effects of physical advection.  Therefore, we propose that the current analysis challenges the popular belief that substantial interday variation in metabolism is the norm.  This hypothesis should be further investigated and the method proposed herein could be used to better understand temporal variation of metabolism at different scales.'}

Table 3: Were the metabolic rates given here those calculated using the authors’ method?

{\it These were based on our methods, without filtering.  Text was added to the footnote for clarity: `...estimated using methods described herein with observed data'}

\end{document}
