\documentclass[letterpaper,12pt,oneside]{article}
\usepackage[paperwidth=8.5in,paperheight=11in,top=1in,bottom=1in,left=1in,right=1in]{geometry}
\usepackage{setspace}
\usepackage{times}
\usepackage{outlines}

%\linespread{2}

\begin{document}

\title{Evaluation of noise related to physical processes in estimates of estuary metabolism}
\author{Marcus Beck}
\date{\today}

\maketitle

\textbf{Points to remember}:
\begin{itemize}
\item What is broader significance, why do we care about this?
\item Megaphone approach - broad to large, i.e., use simple examples to explain concept, then just a logical extension to `all' the data
\item Hourly production estimates during the day can be negative, they are net estimates and do not account for respiration
\end{itemize}

\textbf{Outline}:
\begin{outline}
\1 Introduction
\2 Estimates of ecosystem metabolism - what it is, why we need it, basic methods
\2 State of the science for measuring ecosystem metabolism using open-water method - Caffrey 2004, Caffrey et al. 2013, Staehr et al. 2010, others
\2 Violation of assumptions or complications using Open-water method in estuaries - causes, lit review (Nidzieko et al. 2014), mostly anecdotal to date and site-specific, summary of assumptions in table 7 Staehr et al. 2010
\2 Management concerns in monitoring and ecosystem assessment --  limited resources
\3 Questions of where, frequency, and duration for deployment of sondes (see Staehr et al. 2010) with limited resources
\3 Can we evaluate the extent to which monitoring provides an accurate assessment of actual DO, with relevance for standards/hypoxia criteria?  What if we only sample during periods of increasing tide?
\3 We have x number of estuaries and y $<<$ x water quality monitors, let's put them in set locations for 10 days.  What's wrong with this or what kind of bias could be expected? 
\2 Management concerns in monitoring and ecosystem assessment -- high frequency data
\3 NERRS monitoring -- other extreme, ample data but no noise characterization
\3 Potential approaches to estimate magnitude/direction of bias from existing datasets, contrast with explicit experimental design to answer same question
\2 Study objective, goals, questions
\3 Objective -- Characterize effects of physical processes on DO signal to improve estimates of metabolism with multi-year time series of high frequency water quality data for US estuaries
\3 Can we describe the magnitude of the noise or bias in estimates of ecosystem metabolism?

\3 Can stations be categorized as to the expected types of noise or bias?
\3 If so, can this noise or bias be removed?
\4 For example, what is minimum averaging window or minimum sampling time required to remove noise or bias (e.g., can we reduce the noise to +/- 10\% of expected)?  
\4 Can empirical correction models be developed on a site-level basis that accounts for noise or bias?
\3 The approach is meant to characterize existing noise/bias but also outline strategies for obtaining estimates of wq trends with more accuracy -- What kinds of strategies give us high quality information? 

\1 Methods - Simple to complex
\2 Data source 
\3 NERRS SWMP overview
\3 Data processing (e.g., combination of nut, wx, and wq time series, data qa/qc, filters, which sites were excluded, etc.)
\2 Basic overview and application of open-water method to SWMP data - including air-sea gas exchange estimation via Thiebaut 2008, Caffrey et al. 2013
\2 Emphasis on noise/bias eval, not trends in metab -- this is for another project
\2 Theoretical equation
\begin{equation}
\frac{DO_{obs}}{dt} = \frac{DO_{bio}}{dt} + D + \frac{DO_{adv}}{dt} + \epsilon
\end{equation}
\begin{equation}
\frac{DO_{bio}}{dt} = \frac{DO_{obs}}{dt} - D - \frac{DO_{adv}}{dt} - \epsilon
\end{equation}
\3 Break out advection term into separate components related to season, day/night, spring/neap, etc.
\3 Error term could include any other effects that violate assumptions of method, e.g.,  meteorological effects, assumed to be less important for this analysis but still an issue
\2 Evaluation of potential sources of bias/noise 
\3 Simulated DO advection gradients/tides and solar periods to illustrate effects on DO signal and metabolism
\2 Flowchart description -- situations leading to bias/not all sites are equal
\3 Categorization of time series observation - syn, ant, null 
\3 Exemplary cases but keep it simple (3 or 4 sites) - geom\_ribbon plots and example of phase characterizations, interaction plots of mean inst. DO flux by categories (season, spring/neap, day/night, synchrony)
\2 Characterization of all SWMP sites - for generalization of patterns
\3 Ordination and clustering using tidal components - well-studied, do not overemphasize but describe as basis for characterizing tidal effects
\3 Same methods as above for categorization of obs as syn/ant
\3 Quantification of bias signal by category
\3 Aggregation/averaging over increasing window to remove signal - relate to magnitude of bias/noise, note that aggregation is not ideal
\2 Model for bias/noise correction - obs DO is a function of metab DO, air-sea gas exchange, tidal forcing, and error as above
\3 Estimate relative bias for each station given spring/neap, season, etc. - assumes midpoint between antagonistic/synergistic observations are unbiased
\3 Relative magnitude of bias used as site-specific correction factors
\3 Validate correction model by showing no tidal signal in metabolism after correction


\1 Results
\2 Site-level categorization - ordination/clusters
\2 Types of bias/noise associated with site-level categories
\2 Minimum averaging windows by categories, sites
\2 Functions for removing noise by categories, site 
\3 How well does it work?
\3 Include all as supplementary

\1 Discussion
\2 Recommendations -- results useful for evaluation of high frequency data and use of limited resources
\3 e.g., If site is macrotidal semidiurnal, do not sample less than x number of days
\3 If SWMP, use correction factors
\3 If long-term time series available, id bias and estimate correction factors
\2 Geographic similarity not equal to site-level similarity -- site-specific characteristics could dominate DO signal
\2 Emphasis on tidal advection -- other factors contribute to bias/noise
\2 tidal height not always coincident with tidal excursion
\2 Conclusions

\end{outline}

\end{document}