

load('data/met_comp.RData')

# subset metab estimates by window widths for each case study
load('data/case_grds.RData')
sel_vec <- llply(
  casewins, 
  .fun = function(x) with(case_grds, which(dec_time == x[[1]] & hour == x[[2]] & Tide == x[[3]]))
)
sel_vec <- paste0(names(casewins), '_wtreg_', unlist(sel_vec), '.RData')

met_comp <- met_comp[met_comp$.id %in% sel_vec, ]
In aquatic ecosystems, time series of \ac{DO} have been used to compute estimates of integrated ecosystem metabolism.  Central to this open water or ``Odum'' method is the assumption that the dissolved oxygen time series is a Lagrangian specification of the flow field.  However, most \ac{DO} time series are collected at fixed locations, such that the method must assume changes in dissolved oxygen principally reflect ecosystem metabolism and that effects due to advection or mixing can be neglected.  A statistical model using weighted regression was applied to separate variability in \ac{DO} associated with metabolism from tidal variation or other advection in estuaries, thereby helping to partially relax this assumption and improve estimates of ecosystem metabolism. 
The method targets the periodicity of the tidal component while preserving the true biological signal, offering a distinct advantage over traditional deconvulution methods.  The method was developed and tested using a simulated \ac{DO} time series with known biological and physical components, and then applied to one year of continuous monitoring data from four water quality stations within the \acl*{NERRS}.  Overall, the approach is a useful way to reduce variability in estimates of ecosystem metabolism caused by advection (mean , particularly when the magnitude of tidal influence is high and correlations between tidal change and the solar cycle are low. By reducing the effects of physical transport on metabolism estimates, there may be increased potential to empirically relate metabolic rates to causal factors on times scales of several days to several weeks. 