In aquatic ecosystems, time series of \ac{DO} can be used to infer integrated ecosystem processes such as primary production, respiration, and net metabolism.  However, continuous monitoring data at estuaries may reflect variation from both biological and physical processes, such that observed data may produce inaccurate or misleading process estimates.  Statistical techniques that dynamically quantify variation in \ac{DO} and tidal changes over time have the potential to isolate biological signals in \ac{DO} variation.  A weighted regression method was developed to filter, or detide, the predicted \ac{DO} signal to remove the influence of physical advection on ecosystem metabolism estimates.  The method was tested using a simulation approach to create multiple \ac{DO} time series with known additive components of biological and physical variation on different periods.  The method was further validated using one year of continuous monitoring data at four water quality stations that are part of the \acl{NERRS}. We provide a detailed discussion on use of the method for improving certainty in evaluation of \ac{DO} measurements from sites with strong tidal influences.  Moreover, we propose that the method could expand use of the open-water method for estimating ecosystem metabolism in estuaries given that the approach can produce robust estimates of \ac{DO} that are independent of tidal advection.  In particular, this could facilitate the use of shorter deployment periods for water quality monitors or incomplete time series given that known biases related to water movement could be removed. 