Reliable estimates of ecosystem metabolism depend on measures of \ac{DO} flux that are dominated by biological processes.  Long-term time series of \ac{DO} measurements may include variation related to both biological and physical processses such that the use of observed data may be insufficient or even misleading in some examples.  Statistical modelling techniques that dynamically quantify variation in \ac{DO} over time and tidal changes have the potential to isolate biological signals in \ac{DO} variation to more accurately estimate metabolism.  A weighted regression method that estimates \ac{DO} as a function of time and tidal height was developed to normalize, or detide, the predicted \ac{DO} signal to remove the influence of physical advection on metabolism estimates.  First, a simulation approach was used to create multiple \ac{DO} time series with known additive components of biological and physical variation on different periods.  Comparisons of detided estimates with the known, simulated biological component of the \ac{DO} signal suggested the method accurately and precisely removed varation attributed to tidal advection.  Extension of the method to four case studies provided a proof of concept illustrating the method could be useful for real-world applications. We provide a detailed discussion on use of the method for improving certainty in evaluation of \ac{DO} measurements from sites with strong tidal influences.  Moreover, we propose that the method could expand use of the open-water method for estimating ecosystem metaoblism in estuaries given that the approach can produce robust estimates of \ac{DO} values that are independent of tidal advection.  In particular, this could facilitate the use of shorter deployment periods for water quality monitors or incomplete time series given that known biases related to water movement could be removed with weighted regression. 