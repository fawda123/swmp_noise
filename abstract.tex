In aquatic ecosystems, time series of \ac{DO} can be used to infer integrated ecosystem processes such as primary production, respiration, and net metabolism.  However, \ac{DO} time series data at estuaries may reflect variation from both biological and physical processes, potentially leading to inaccurate or misleading ecosystem process estimates.  One such physical process is the occurrence of large lateral \ac{DO} gradients in an estuary which may advect water with different \ac{DO} characteristics past a sensor.  In such situations, the lateral gradient may cause variation in \ac{DO} time series that are not attributable to metabolic processes.  Statistical techniques that dynamically quantify variation in \ac{DO} and tidal changes over time have the potential to isolate biological signals in \ac{DO} variation.  A weighted regression method was developed to filter the \ac{DO} time series to remove the influence of physical advection, thereby removing bias or noise in ecosystem metabolism estimates.  The method was tested using simulated \ac{DO} time series with known additive components of biological and physical variation.  The method was validated using one year of continuous monitoring data at four water quality stations that are part of the \acl{NERRS}. We provide a detailed discussion on use of the method for improving certainty in ecosystem metabolism estimates from sites with strong tidal influences.  This approach could improve metabolism estimates using shorter deployment periods or incomplete time series by removing bias attributed to lateral water movement. 